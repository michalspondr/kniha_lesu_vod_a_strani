\documentclass{book}
\usepackage{verse}
\usepackage{romannum}
\usepackage[czech]{babel}
\usepackage[utf8]{inputenc}
\title{Kniha lesů, vod a strání}
\author{Stanislav Kostka Neumann}
\renewcommand{\poemtitlefont}{\raggedright\normalfont\large\bfseries\hspace{\leftmargin}}
\begin{document}
\pagenumbering{arabic}
\maketitle
\tableofcontents
\newpage
% nechci zobrazovat Obsah na každé stránce
\markboth{}{}
\poemtitle{Vstupní modlitba}
\begin{verse}
Ve jménu života i radosti i krásy.
\end{verse}
\begin{verse}
Hle, země naše, ty, jež ležíš pod nebesy\\
jak žena kvetoucí pod zrádným závojem,\\
buď svato jméno tvé všem lidem po vše časy,\\
přijď nám tvé království se všemi svými plesy,\\
nás ponoř v příval svůj a zajmi sladkým snem.
\end{verse}
\begin{verse}
Buď vůle tvá nám vším, jak ptáku je a hmyzu,\\
pokorně bylině i zpívající vodě,\\
jež z drobných pramenů chce míti veleproud;\\
tvá vůle prostup nás jak uhel žíla kyzu,\\
abychom žili s ní ve světlé, moudré shodě\\
a s jasnou hrdostí tvým rodem chtěli slout.
\end{verse}
\begin{verse}
Vezdejší chléb svůj si již dobudeme sami,\\
když máme času dost na paláce a básně,\\
na lesklé sítě drah, sny, věže, kabely;\\
však síly třeba nám, jež zrušila by klamy\\
a hlucha k skuhrání klad žití zdvihla jasně\\
i naše synovství nad zápor zbabělý.
\end{verse}
\begin{verse}
A viny odpusť nám, jež nevědomosti plodí,\\
jichž dračí semeno do prostných srdcí sejí\\
sluhové fantómů a blasfemických věr.\\
Jsme děti svedené; jež bludičky nás vodí\\
do bahen ohavných, že ve své beznaději\\
ti, matko, kynem pak pro jih i pro sever.
\end{verse}
\begin{verse}
Však do pokušení nás uveď v každé chvíli,\\
vše chceme okusit, čím tělo tvé nám kyne,\\
kypící, milostné a širé tělo tvé!\\
My žádostivost svou z tvých mocných ňader pili,\\
tvá míza v poskoku se cévami nám řine\\
a lačných útrob tvých jsme květy žíznivé.
\end{verse}
\begin{verse}
Jen silné učiň nás ve víře, v lásce k tobě,\\
a jak hvozd na jaře se obrodí náš rod;\\
v temnosvit života se pohrouží jak robě\\
pro sladkou zralostí již pukající plod.\\
Tak zlého zbavíš nás jak černé snětí klasy\ldots
\end{verse}
\begin{verse}
Ve jménu života i radosti a krásy.
\end{verse}
\newpage
\poemtitle{Prolog}
\begin{verse}
Nesmrtelnou hymnu slíbil jsem kdys lesům.\\
Snadno přísahy se činí v líbánkách.\\
Slíbil jsem ji stromům, zvěři, hmyzu, vřesům.\\
Milovali jsme se v rozkošnických snách.
\end{verse}
\begin{verse}
Nesplnil jsem slibu, nesplnil ho ani,\\
když ta naše láska v štěstí uzrála.\\
Já jen o tom našem věrném milování\\
zpívám prosté rytmy lesů vazala.
\end{verse}
\begin{verse}
Také v jejich stínu střídá se sled chvíli;\\
tisíc bylo žalů, tisíc radostí:\\
Žaly, moje žaly hymnu pohltily;\\
mír a radost rády píseň pohostí.
\end{verse}
\begin{verse}
Z radostí a míru vzcházejí mé sloky,\\
smutků překonaných bývá na nich pel\ldots\\
Unikl jsem z města mílovými kroky,\\
k zdroji věčné něhy blah jsem odešel.
\end{verse}
\newpage
\poemtitle{Ocúny}
\begin{verse}
Ocúny na lukách, žluté skvrny v lesích,\\
má novou svěžest poslední tráva.\\
Umřeme v barvách, umřeme v plamenech.\\
Pak zatopí nás bělostná láva.
\end{verse}
\begin{verse}
S večerem mlhy sklouzají po lesích;\\
jak zakletý zámek tichá je řeka.\\
Mír mimo dobro a zlo se klade\\
na křídla ducha zavřená a měkká.
\end{verse}
\begin{verse}
Kdes v dáli šumí červnové vášně\\
ozvěna sladká, nesmírně táhlá.\\
Do rosy klidně složila hlavu\\
květina sluncem srpnovým zpráhlá.
\end{verse}
\begin{verse}
Ocúny na lukách, žluté skvrny v lesích.\\
Umřeme v barvách na zcela malou chvíli.\\
Básníci života, jsme jako země:\\
zas rozkvetou růže, jež jsme zasadili.
\end{verse}
\begin{verse}
Na jednu zimu zapomenuti,\\
k novému jaru z mrtvých povstaneme.\\
Ocúny na lukách, žluté skvrny v lesích:\\
od ocúnů k prvním sněženkám jdeme.
\end{verse}
\newpage
\poemtitle{Září}
\begin{verse}
Ty, sladký září! Jak je modré nebe\\
nad těmi vrchy, kde se barvy rodí!\\
Jdu zmaten vzduchem, jenž je plný tebe,\\
a chtěl bych plouti povětrnou lodí.
\end{verse}
\begin{verse}
Tak zcela nízko nad lesy, jež mění\\
svou píseň zelení na píseň žlutí,\\
a v slunci tiše oddati se snění\\
o novém jaru, novém zahynutí.
\end{verse}
\begin{verse}
I tak je dobře však na zemi něhy,\\
jež dovede tak krásně umírati\\
na chvíli, požár uhašený sněhy,\\
by znovu vzplála s jarem, jež se vrátí.
\end{verse}
\begin{verse}
Již v lesích strání měď a zlato hoří,\\
koňadra u cest nepokojně hvízdá\\
v klid, jenž se snáší ze sivého boří\\
a v němž svou radost poslední teď hmyz dá.
\end{verse}
\begin{verse}
Motýla zdvíhám létem znaveného\\
a zas ho pouštím na poslední květy;\\
ještě se těšit budem ze dne svého\\
já, cikády a ptáci, zvěř a květy\,\ldots
\end{verse}
\begin{verse}
Proč prchají však světle hnědé srny?\\
Můj krok přec tišší nemůže už býti.\\
Jsem tulák trochu rozedraný trny\\
a teplo zvířat chtěl bych pocítiti!
\end{verse}
\newpage
\poemtitle{Oběť poděkovací}
\begin{verse}
Těch ohňů podzimních a jejich namodralých kouřů,\\
jež rozptylují se tak příliš pomalu!\\
Jsou jako modlitby či oběť poděkovací\\
za všecky rozkoše letního zápalu.
\end{verse}
\begin{verse}
Za mladá těla žen a jejich supějící vášeň,\\
za slunce skvělý dar od června do září,\\
za lesů vonný stín a louky rozezpívané,\\
za květy, ptáky, vody, hudbu komáří\,\ldots
\end{verse}
\begin{verse}
A mráčky kouře děkují\ldots A nevědouce komu\\
do lesů tiše plynou, kde již usíná\\
pod rzí a krví stromů skála znovu tvrdnoucí,\\
jež slední vzdech svých ňader pevně upíná\ldots
\end{verse}
\begin{verse}
Poutníče, zvěstuj v nížinách a v sazovitých městech,\\
my že tu šťastni byli s lesy po horách,\\
že také děkujeme slunci\ldots zemi chladnoucí\ldots\\
za život pohanský, sen v barvách, lesklý prach\,\ldots
\end{verse}
\newpage
\poemtitle{Dubisko padlo}
\begin{verse}
Za mostem u Myší díry jednoho rána v září\\
veliké dubisko padlo z úpatí skalnaté stráně,\\
na louku zrosenou padlo k slunci tíhnoucí tváří;\\
dělníků osm tu stálo, listí pršelo na ně.
\end{verse}
\begin{verse}
Sekyrou podťato hlučně a pilou podřezáno\\
do trávy zrosené padlo, až v kučeravé hlavě\\
tisíce stonů zavzdychlo a zapraštělo v ráno,\\
jež rez a žluť a červeň strání odhalovalo právě.
\end{verse}
\begin{verse}
Svitava vzdorně hučela, do balvanů bila,\\
strhujíc v bělostnou pěnu zlaté habrové listí,\\
nad černavými bory kavka se rozkroužila:\\
dělníků osm tu stálo, lidé, jimž se chce jísti.
\end{verse}
\begin{verse}
Dělníků osm tu stálo z rozkazu knížete pána,\\
nepřátel osm tvrdých vtrhlo sem z blízké vísky,\\
v samotu, v ticho za řekou: v šest padla první rána;\\
za dubem padnou duby, habry, břízy a lísky.
\end{verse}
\begin{verse}
Na sta jich padne, útlé i okoralé již kmeny,\\
v bok stráně vnikne kopáč, až jiskry vydá kámen,\\
balvany zřítí se v řeky tok hlučný a rozpěněný,\\
mech v hlíně zalkne se udupán a hraboš prchne zmámen.
\end{verse}
\begin{verse}
To cestu rozkázal budovat náš katolický pan kníže;\\
do Dlouhých strání se povine vzhůru tišinou lesů,\\
by z rodné hlíny v údolí a k nádraží měly blíže\\
mrtvoly smrků a sosen padlých v bitevním děsu.
\end{verse}
\newpage
\poemtitle{Tam, kde město počíná}
\begin{verse}
Ze starých topolů tu prší srdce sivá,\\
do řeky, zkažené mdlou špínou předměstí,\\
v níž továrenský kal a barva nepravdivá\\
zpěv zkouší duhový o shnilé neřesti.
\end{verse}
\begin{verse}
Tok teplých odvarů a splašek ustálených\\
lesklými žilami mdlá žíhá mastnota;\\
ryb těla stříbrná od břehů prchla zděných\\
plesnivým kamenem, v němž bahno klokotá.
\end{verse}
\begin{verse}
Jak zpustlý hřbitov tu i dno je zneuctěno,\\
kde mrtvá koťata se jistě válejí;\\
puch stoupá soumrakem, a vše je zachmuřeno,\\
když tudy otroci jdou domů z galejí.
\end{verse}
\begin{verse}
A umírání sil je tady teprv smutné,\\
tak jako renouveau je k pláči žalné tu;\\
jen zima přijde sem jak smilování nutné\\
a hanbu ukryje v svém bílém sametu
\end{verse}
\begin{verse}
snad aspoň na dnů pár, ač neuvěříš ani,\\
že také tenhle kout se někdy rozjasní\ldots\\
Och, jak je krásné teď tam u nás umírání,\\
kde vrchy planoucí o zašlém létu sní
\end{verse}
\begin{verse}
a všechen slunce jas, jejž dlouhým douškem pily,\\
teď v nach a ve zlato svých lesů vdechují -- \\
och, jak je svěží teď a jasně rozpustilý\\
tam u nás řeky proud, s nímž vlny strhují
\end{verse}
\begin{verse}
od rodných břehů vrb a olší listí sváté\ldots\\
Zde je vše prokleto a zhanobeno však,\\
vzduch, půda, řeka, strom i dětí tělo zlaté\\
i oprýskaná zeď, jež ční jak ztvrdlý mrak.
\end{verse}
\begin{verse}
A místo pryskyřic a lesů bílé páry\\
již z dlouhých komínů dým stoupá k blankytu:\\
před sluncem ukrývá své zmrzačené tvary\\
kout země zchátralé a zmírající tu.
\end{verse}
\newpage
\poemtitle{Na podzimním slunci}
\begin{verse}
Obelstěn sluncem, vzešlým opožděně\\
nad dohořívajícím požárem,\\
jejž podzim zažehl, jak zbité štěně\\
nějaké rány léčit krásným dnem\\
se vleku na výsluní mezi lesy\\
a k pařezu dím: Odpočineme si.
\end{verse}
\begin{verse}
Sám věru nevím, oč tu vlastně běží,\\
proč duch i tělo zmalátněly tak,\\
proč bolest v nitru nevrle se ježí\\
a smutek plíží se jak polem svlak,\\
jenž stonky obepíná, dusí květy\\
a jehož zbytky vždy jsou nevyplety.
\end{verse}
\begin{verse}
Sám věru nevím, kterou zlatým jitrem\\
teď léčím ránu. Stero proniká\\
střel otrávených otupělým nitrem.\\
Jeť žalostný dnes osud básníka\\
ve vlastech, kterým k prostitutek smíchu\\
jsou mladá srdce zpívající v tichu.
\end{verse}
\begin{verse}
Snad muž, snad žena, dav snad, země celá,\\
och, což já vím, kdo nejvíc poranil!\\
Ze stráně, která včera dohořela,\\
se třesu běláskem, jenž v říjen zbyl,\\
a klamným sluncem dávám obelstíti\\
den jeden svůj pro anemické kvítí.
\end{verse}
\newpage
\poemtitle{Pytláci}
\begin{verse}
Říjnové ráno dozrává již v mlze, tichu a chladu\\
nad řekou hustou a olověnou, jež zase trochu klesla,\\
kolkolem jak by se rozptylovala voda v dychtivém hladu,\\
na vodě naše pytlácká loďka: šplouchají červená vesla.
\end{verse}
\begin{verse}
Napravo, nalevo po stěnách strání sotva se tuší lesy,\\
jež zcela jistě z mlhy dnes vyjdou holejší zas a teskné;\\
v šedivém vlhku váhavě plujeme hluboko na dně kdesi,\\
vetřelci, na něž se ryba jde dívat a na hladinu pleskne.
\end{verse}
\begin{verse}
Šplouchají vesla, mlha se válí, vrána pozdravuje\\
den plížící se do zšedlých borů i do hnědých dubin.\\
Po vodě naše pytlácká loďka jako ve snách pluje,\\
z vrch sivých prchají vodní žínky do neprůzračných hlubin\ldots
\end{verse}
\begin{verse}
A již se cos hnulo. Stoupají mlhy po vrších z vodního klínu\\
do lesů nahoru, oblaka šedá mezi nebem a námi,\\
z vrcholu stráně k vrcholu naproti klenutí světlých stínů,\\
tragická vteřina před bitvou se sluncem, jež vstalo za horami.
\end{verse}
\begin{verse}
Na vodě naše pytlácká loďka\,\ldots \uv{Chval každý duch Hospodina!}\\
Blýsklo se za námi nahoře na vrchu! Z borů se vyhoupl náhle\\
stříbrný kotouč tajemně svítící v mlhy oblaka siná,\\
a v dáli jako by zazněly polnic fanfáry táhlé.
\end{verse}
\begin{verse}
% FIXME "položme" velkým?
Hej, holá! položme červená vesla, na kořist nemysleme\\
a sepněme ruce v modlitbě tiché k tomu, jenž líbá zemi\\
a jemuž tak skoro za vše, co máme, za život děkujeme\\
\ldots i za tu rybu, jež se třepe snad chycená pod olšemi.
\end{verse}
\newpage
\poemtitle{Se složenými vesly}
\begin{verse}
\textit{\ldots es durchweht mit ein Erkennen,\\
wie grenzenlose Weiten Meschnen trennen,\\
wie furchtbar einsam unsre Seelen leben\ldots\\
H. von Hofmannsthal}
\end{verse}
\begin{verse}
Složil jsem vesla. Po vodě loďka kolébá se a plyne.\\
Je řeka tu klidná jak hluboký rybník mezi blízkými břehy,\\
kde vrbové pruty k olšové větvi chmelová liána vine,\\
kde z šedého stříbra vrbových houštin list padá jak slza něhy.\\
Složil jsem vesla. V soumraku, v mlhách podzimní píseň hyne.
\end{verse}
\begin{verse}
Hladové noci nám upíjejí malátný půvab denní,\\
jenž velké své štěstí v slunečním jasu bez tepla poznává stěží,\\
po lesích strání plameny hasnou v zimomřivém chvění:\\
bůh ví, kde ptáci jsou, kde květin všechna semena leží.\\
Složil jsem vesla. V soumraku, v šedi život je zapomnění.
\end{verse}
\begin{verse}
V mokvavém chladu na loďce sedím\ldots jako by nebylo lidí\\
za těmi mlhami, za těmi lesy, jako bych tady byl doma\\
pod olšemi k vodě skloněnými, kde parma vousatá slídí,\\
srn, jež jdou pít sem, ryb a kavek bratr s vyschlýma rtoma.\\
Myslím si: Zasil jsem, ať kdo chce, co kde chce a pro kohokoliv sklidí.
\end{verse}
\begin{verse}
Od divokých břehů šílených měst přišel jsem k vodě a lesům,\\
já, který se učím teď chápati jich nedůvěřivost k lidem,\\
já, který se naučil rozuměti jejich nejmenším hlesům\\
a daleko lidí býti šťasten i trpěti s jejich klidem;\\
svých včerejšků neznám a svoje zítra zpívám na vrších vřesům.
\end{verse}
\begin{verse}
Od člověka k člověku nesmírná cesta a ještě nedojdeš k cíli;\\
se stromy a balvany, s travou a řekou, se zvěří, s hmyzem jsem jedno.\\
Chci, aby bory, louky a vody voněly z dnů mých a chvílí:\\
jsem zde teď, haluzka domácí půdy, hrouda a já jsme jedno,\\
pro sebe jsme se narodili, abychom sobě žili\ldots
\end{verse}
\begin{verse}
Složil jsem vesla. Stíny se stýkají. Jak je všechno krásné!\\
I pomalé zmírání znavených krajů pro zimní klid a spánek,\\
jenž přichází se slibem širých ploch, jež bělostné jsou a jasné.\\
Loďka se kolébá. Hladinu čeří po proudu studený vánek.\\
Zde šťastný život pohádkou není, zde se šťastně i hasne.
\end{verse}
\newpage
\poemtitle{Listopad mezi buky}
\begin{verse}
Všechny jízvy strání se již obnažily\\
zvětralé a šedé, mechem skvrnité.\\
Suchá voda větru šumí steskem lesů,\\
po vrších je slunce měkce rozlité.
\end{verse}
\begin{verse}
Slunce milosrdné, jež by chtělo hřáti\\
stromů vrcholky již polobezlisté\ldots\\
chladný den má vůni hrdé rezignace,\\
barvy rozkladu a lesky zlatisté.
\end{verse}
\begin{verse}
Bloudím žleby mezi bukovými lesy.\\
Ještě mají stromy trochu zeleně,\\
vydechující do spleti kmenů choře\\
lehce fialovou páru jeseně.
\end{verse}
\begin{verse}
Bloudím tiše s křídly klidně složenými,\\
zhnědlé zlato buků šustí pod nohou,\\
na dně žlebu ručej zvoní ledovitě,\\
ptáci -- touho! -- ptáci zpívat nemohou\ldots
\end{verse}
\begin{verse}
Stojím v klíně kopců; harmonií stesků\\
sladce dýše píseň listopadová.\\
A mne mrzí jedno: v černém haveloku\\
že tu stojím jako skvrna surová.
\end{verse}
\begin{verse}
Raděj byl bych faunem chlupatým a hnědým,\\
v hnědých proudech listí sotva zřejmý bod:\\
na bukový pařez s píšťalou bych used\\
listům padajícím hráti doprovod.
\end{verse}
\newpage
\poemtitle{Doma}
\begin{verse}
Mlhy jsou v kotlinách a bez lesků sivé je nebe;\\
vejde-li člověk do lesů, hlavu zadumán svěsí.\\
Po rusých temenech vrchů zavřených v sebe\\
na šedých šlářích ticho zkřehlými pluje lesy.
\end{verse}
\begin{verse}
Datel jen odklepává hubený den svůj a těká\\
po sosnách fialových. Sýkora v habří pískne.\\
Melancholická radost a barev dřímota měkká\\
pod nebem uzavřeným k zemi přátelsky tiskne.
\end{verse}
\begin{verse}
V teplém tak hnízdě ptáku dobře a veselo bývá,\\
hledí-li mřežovím větví z rodného stromu\ldots\\
Obloha má ovšem slunce, ale je nepravdivá.\\
K zemi se navracíš. I když je chudá. Domů.
\end{verse}
\newpage
\poemtitle{Vločky jdou}
\begin{verse}
Zatáhla se nebesa těžkou šedí sněhovou,\\
milióny vloček jdou, rozsypou se: hou, hou, hou,\\
se zemí se nebesa spojí sítí ledovou,\\
země ztratí špinavá unavenou barvu svou.
\end{verse}
\begin{verse}
Radosti své rozloží bílé plochy veselé,\\
neubrání se jí ni černé bory na horách,\\
a jen řeka kalná, mdlá jako z tuhy vyvřelé,\\
zhltne vločku za vločkou, světélka, jež mizí v tmách.
\end{verse}
\begin{verse}
Spadne k zemi velký mír s šumem skoro neslyšným,\\
\uv{odpusťme si, co jsme si,} šeptly vrchy, pole, les;\\
bílé ticho ulehne se životem bezdyšným\\
v širý kraj a širý klid vanoucí až do nebes.
\end{verse}
\begin{verse}
Krtek spí již v doupěti, v hroudě strnul drobný hmyz,\\
zrno jasně vzklíčené čeká povlak sněhový,\\
holé stromy svírají zastavené proudy míz,\\
skála o svých nadějích ničeho již nepoví\ldots
\end{verse}
\begin{verse}
Přes noc spadne bílý div do polí a do lesů,\\
bílý mír a bílý jas, pokoj všemu stvoření.\\
K zásypům svým půjde zvěř, v oku klid a bez hlesu,\\
v hluboký a tichý žleb, kde se vzpomíná a sní.
\end{verse}
\begin{verse}
Sní a čeká\,\ldots Člověk jen v děrách svých v ulicích\\
hladem, krví, mozkem štván šílenství svých nezmění;\\
v bílý den i v bílou noc řičí jeho řev i smích,\\
neustálý boj o život, neustálé říjení.
\end{verse}
\newpage
\poemtitle{Vysoko uprostřed lesů}
\begin{verse}
Vysoko uprostřed lesů stojím o šedé dubisko opřen,\\
přišel jsem z bláta dědiny sem, kde třpytí se bílý sníh,\\
kde křesťanů bůh je neznám zcela a každou haluzí popřen,\\
kde nad čerstvou stopou zvěře táhne dech míru po vrších.
\end{verse}
\begin{verse}
Napravo, nalevo, dokolo kolem mřežovím kmenů a větví\\
na ztemnělé moře vrchů vidím, kde u lesů stojí les;\\
jeho rozloze klidné a hrdé jen dravého ptáka let ví,\\
o jeho hloubce liška snad zaštěká z mlh ranních do nebes.
\end{verse}
\begin{verse}
U vlny černozelené vlna, která je černohnědá,\\
ční nepohnutě a mlčky z moře jak v zakletí a snách;\\
nad nimi visí se slibem sněhů obloha bez konce šedá:\\
uprostřed malý-veliký stojím tu: květ kvete mi na vlnách.
\end{verse}
\begin{verse}
Ze spící skály a nad bílé sněhy pomněnkově kvete,\\
nahoře v lesích jej opatruji a vlastní mou krví je živ,\\
substance krve mé dává mu lesky, tvar jeho měkce hněte;\\
země a člověk krvesmilnili a zplodili modrý div.\\
\end{verse}
\begin{verse}
Zplodili o jedno štěstí více, upřímné, veselé, prosté,\\
jak zrno z hroudy vyňaté sluncem, jež směje se na líchu;\\
v září mi uzrálo uprostřed plamenů, nad sněhy teď mi roste,\\
vod, hlíny, medů a pryskyřic má vůně v kalichu\,\ldots
\end{verse}
\newpage
\poemtitle{Leden}
\begin{verse}
Nad sněhovými plochami spí černých koster lesy\\
po stráních, které zbělely a nesou nahé jizvy\\
ve svit, jenž stoupá od země a v šeď ztlel pod nebesy.\\
%FIXME Tu poutník,
Tu poutník když se objeví, jak ruka zpupné výzvy
\end{verse}
\begin{verse}
v tvář ticha, spánku, osudu ční nad studené sněhy.\\
%FIXME čárka za smrti
Den krutý je a přísnější než pocel samé smrti\\
jež v lůně nových životů vždy slib má plný něhy --\\
den krutý je a každý hlas bílými spáry škrtí.
\end{verse}
\begin{verse}
A mezi ledy břehů svých tok řeky zeje mrtvý,\\
tajemnou hloubkou hedvábnou v chlad svého ticha láká\\
jak propast táhlá, řekl bys, o které tu jen smrt ví.\\
Však černá voda zrcadlí vrb těla křivolaká.
\end{verse}
\newpage
\poemtitle{Zimní noc}
\begin{verse}
To není země, to je sen,\\
jejž bledý měsíc vykouzlil\\
pod zastíněnou oblohou\\
na vlnách sametových mil.
\end{verse}
\begin{verse}
To není země, to je div,\\
nesmírná hudba bílých cest,\\
jež pro své tiché jiskření\\
do hlubin strhly světlo hvězd.
\end{verse}
\begin{verse}
To není země, to je zjev,\\
jenž vyplul z nekonečnosti.\\
Já, blázen, bod a cizinec,\\
jdu směšný mraznou věčností.
\end{verse}
\newpage
\poemtitle{Prosté sloky}
\begin{verse}
\large{\Romannum{1}}
\end{verse}
\begin{verse}
Miluji hvězdná nebesa\\
pro jejich hloubku a krásu,\\
pro jejich modrou záhadu\\
plničkou třpytného jasu.
\end{verse}
\begin{verse}
Napohled úsměv, ticho, mír,\\
ve skutečnosti strž světů\\
nejhezčí, nejpodivnější\\
bez bohů a jejich tretů.
\end{verse}
\begin{verse}
Stanul jsem v noci lednové\\
uprostřed nesmírných sněhů:\\
velebnost hmoty chápal jsem\\
a v srdci pocítil něhu.
\end{verse}
\begin{verse}
\large{\Romannum{2}}
\end{verse}
\begin{verse}
Nevíme kam a nevíme proč,\\
záhady všude je tolik;\\
nebesa hvězdná podivná jsou,\\
podivný hvězdnatý dolík,
\end{verse}
\begin{verse}
tak jako každý života děj,\\
černých těch borů tam zrání --\\
i to, že dřepič z kaváren, já,\\
šťasten jsem na sněžné pláni.
\end{verse}
\newpage
\poemtitle{Ojíněly lesy}
\begin{verse}
Ojíněly husté smrky, dlouhé borovice,\\
zkřehle věsí jehličí své, jak pod bílou plísní;\\
sněžné skvrny příliš hutně imitují slunce,\\
sýkor pískot nesmělý chce býti možná písní.
\end{verse}
\begin{verse}
Ojíněly kostry habrů, modřínů a dubů,\\
na mlázi jen hnědé listí schlíple ještě visí;\\
velké ticho dříme v mlze měkké, šeré, vlhké,\\
která s jemným steskem věcí důvěrně se mísí.
\end{verse}
\begin{verse}
Ojíněla stébla trávy v sněžnou vegetaci,\\
jež se láme pod kročeji, neduživě praská,\\
velcí ptáci vzletí tiše, mizí v srdce lesů:\\
samota mě svými prsty hedvábnými laská.
\end{verse}
\begin{verse}
Je v tom trochu život a trochu vlídné smrti,\\
odhozené iluze a nadějí svět celý.\\
Sentimentální by byla žena tudy jdoucí,\\
ale básník slyší všude spodní akord vřelý:
\end{verse}
\begin{verse}
%FIXME sílu
důvěru a klidnou silu v životi i smrti,\\
pod melancholickou tváří v krásný osud víru\ldots\\
V chladu stesků ojínělo také jeho srdce,\\
ale v hlubině zní píseň pohody a míru.
\end{verse}
\newpage
\poemtitle{Spánek únorový}
\begin{verse}
Mlhavý den únorový, hnědá převládá.\\
Poprašek, jenž v noci spadl, tvoří trochu skvrn,\\
bílé kontury a stezky. Pustá nálada\\
vane nad stopami kavek, nad stopami srn.
\end{verse}
\begin{verse}
Lesy jako vyhořelé dechu nemají,\\
mrtvá hnízda obnažená rozpadají se,\\
všechen život jako když se v pupen utají\\
zakletý a spící pevně v tvrdém obryse.
\end{verse}
\begin{verse}
Ztuhlou zemi zšedlá tráva měkčí nečiní,\\
ani mechy vysílené, listí vyrudlé;\\
zvěř a pták jsou pouhé stíny v šeré jeskyni,\\
jež se plíží tichem vrchů jako po truhle -- -- --
\end{verse}
\begin{verse}
Nenoste sem rudých ohňů velkých pochodní!\\
Nenoste sem vřesku polnic, větrů šílení!\\
Je to maska, spadne náhle, světy zavodní\\
pod úsměvem modrých nebes proudy zelení.
\end{verse}
\newpage
\poemtitle{Noc přípravy}
\begin{verse}
Srdce, slyšíš? Duní to a ječí\\
v temnou noc, jež mlhou napojena.\\
Řeka vzepjala se náhlou křečí,\\
k porodu se chystající žena,\\
slavné dění počlo za horami,\\
kde se tká již úsměv ze zlata,\\
z noci severní vstříc jde mu tmami\\
puklých ledů kantáta.\\
Srdce, slyšíš? Buší na tvé stěny.\\
%FIXME Otevři bylo původně
Čas je, otevři se, než den vzplá.\\
Přede dveře postav na stráž feny,\\
aby člověčina zpozdilá\\
nezasedla trůnů, které patří\\
bohům lesů, žlebů, luk a vod:\\
Až mé oči první zeleň spatří,\\
už ti neublíží svod!
\end{verse}
\newpage
\poemtitle{Březen}
\begin{verse}
Ještě chvilku jen! A moje vděčnost\\
opět pozdraví tvoje milosrdenství.\\
Lidé ciframi měří tvou užitečnost,\\
čím však mně jsi, mé srdce ví.
\end{verse}
\begin{verse}
Země, zítra se zazelenáš!\\
Země, zítra nám rozkveteš!\\
Nikdo nepodplatí tě, nekoupí za otčenáš,\\
ale věrnému synu vším se zveš.
\end{verse}
\begin{verse}
Zpěvem, letem a hnízděním ptáků,\\
vůní, ztepilostí a barvami bylin svých,\\
víc než zlatem klasů mně rudostí máků\\
tisíc radostí učiníš nezasloužených.
\end{verse}
\begin{verse}
Šerou pohádkou vody, luk šťavnatostí,\\
lesů ševelem, zvěří a pryskyřicemi\\
osvobodíš mě z bídy, učitelko ctností,\\
dechem života naplníš plíce mi.
\end{verse}
\begin{verse}
%FIXME verunka bylo původně
Nedočkavá Verunka v sednici hledá kvítí.\\
Spolu toužíme, březnovou nemocí stůněme.\\
Lidem nemajíce zač vděčni býti,\\
celou vděčnost svou, země, vstříc ti neseme.
\end{verse}
\newpage
\poemtitle{U Děravé skály}
\begin{verse}
Na šedém balvanu Děravé skály\\
se samotou srůstaje sedím,\\
pode mnou vody jak by se rvaly,\\
na zběsilou Svitavu hledím.
\end{verse}
\begin{verse}
Rozbouřená písnička vody,\\
mé mladosti, mých nadějí parafráze,\\
podjarní předzvěst nové budoucí shody\ldots\\
jí naslouchám a je mi blaze.
\end{verse}
\begin{verse}
Pleskají, hučí, šumí a zvoní vody,\\
žlutavé vlny letí, rozbíjí se a pění,\\
vzduch revolučními chvěje se svody;\\
však lesy strání dumají v snění.
\end{verse}
\begin{verse}
Jakož jim kážou zákony rodné hlíny,\\
oddaně mízy vzestup očekávají,\\
své věrny zemi, věrny sobě pod oblačnými stíny\\
pevně se drží a zrají.
\end{verse}
\begin{verse}
I já tu přitisknut ke skále chladné\\
své půdě i sobě věrnost přísahám celou,\\
bez lítosti nad tím, co ve mně vadne,\\
tomu, co klíčí ve mně, oddanost slibuji vřelou.
\end{verse}
\begin{verse}
Svoboden býti, toť kvésti a zráti\\
po zákonech svého růstu,\\
neohlížet se napravo, vlevo nešilhati,\\
nevěřit hýření, neřkuli půstu.
\end{verse}
\begin{verse}
Na věky hotov s bohem, s osudem usmířený,\\
marnými blasfémiemi nedráždím nervů;\\
miluji slunce, zemi, lesy, vody a ženy --\\
s lidmi se, když je to nutné, bez bázně servu.
\end{verse}
\begin{verse}
Jen abych nezradil sebe, toho jest nejvíce dbáti,\\
bytosti rytmus sblížiti s rytmem země,\\
na pevné půdě co nejpevněji státi\\
a vteřinu pochopit jemně\ldots
\end{verse}
\newpage
\poemtitle{Jarní zvěstování}
\begin{verse}
Tak dlouho čekali jsme rozechvěni touhou\\
a chladem, který vál ze sněhů svítících,\\
den žití chtěli jsme dát za sněženku pouhou,\\
však blesky chladnými jen vysmál se nám sníh.
\end{verse}
\begin{verse}
Až náhle začlo tát a vzedmula se řeka,\\
divadlo veliké pod lesy hrály kry;\\
my zřeli s rozkoší, jak dole proud se vzteká,\\
a připravovali své jarní mimikry.
\end{verse}
\begin{verse}
Však březen závistně nám nové poslal sněhy,\\
sen sněženkových spoust nade vším rozestřel;\\
tou bílou pohádkou, již z hebké utkal něhy,\\
my šli jsme zmateni a v srdcích měli žel.
\end{verse}
\begin{verse}
%FIXME květ, co
A neupřímný květ co jitro sněžil venku,\\
by slunci vzdoroval, jež přišlo k poledni;\\
kůl každý v plotě sníh měl skvělou za čelenku;\\
však my jen čekali na sněhy poslední.
\end{verse}
\begin{verse}
Tak dlouho čekali jsme touhou rozechvěni.\\
Dnes\ldots atmosféra jest jak brána dokořán\\
ve svěžest, modro, jas a jitřní kuropění,\\
dnes první sladký vznět byl země lůnu dán.
\end{verse}
\begin{verse}
Zdrávas, země,\\
plná milosti,\\
požehnaná jsi mezi květy,\\
země,\\
požehnaný plod života tvého,\\
člověk!
\end{verse}
\begin{verse}
Cos méně zřejmého než nádech neurčitý\\
a jistějšího přec nad všechnu jistotu,\\
v kraj spánkem znavený se sune s mdlými svity\\
a klade první šlář na jeho nahotu.
\end{verse}
\begin{verse}
Již ptáku moudrému po milence se stýská,\\
již rouchou svatební má pěnkava i drozd;\\
ze sterých hrdélek změť tónů ryzích tryská:\\
za sedmi řekami spí chladná minulost.
\end{verse}
\begin{verse}
Však naší řeky proud již klesající zvolna\\
ke ptačím písničkám svůj zpívá doprovod,\\
jenž stoupá po stráních jak hudba sladkobolná,\\
jak snění dívenky, jež v první klesla svod.
\end{verse}
\begin{verse}
Ve skrytých poupatech se rodí cudné tvary,\\
proud barev mladistvých již čeká v pupenech;\\
by v tělech roznítil nám sladce tajné žáry,\\
na každé haluzi se jitří vonný dech.
\end{verse}
\begin{verse}
To dekor slavnostní se pro dny snubní staví,\\
pro hříčky milostné se stelou koberce\ldots\\
Jak lačné kalichy své vztyčujeme hlavy\\
a vůně omamná nám stoupá ze srdce.
\end{verse}
\begin{verse}
Zdrávas, země,\\
plná milosti,\\
požehnaná jsi mezi světy,\\
země,\\
požehnaný plod života tvého,\\
člověk!
\end{verse}
\newpage
\poemtitle{Jiné renouveau}
\begin{verse}
Rozhučela se Svitava kalná a rozvodněná,\\
že břehy nestačí nikterak špinavě hnědému proudu.\\
Své jehnědy z vody zdvíhají keřiska potopená.\\
Záchvěvy roztávání vzrušují zmrzlou hroudu.
\end{verse}
\begin{verse}
Rozhučela se Svitava kalná a rozvodněná.\\
Je ve vzduchu naděje plno a cudných paprsků března,\\
jak by mi po boku kráčela s nesmělou láskou žena,\\
jež tuší, že dnes první prchavé polibky sezná.\\
A před lesů tváří tichou a jakoby lhostejnou ještě\\
teď vzpomínám si, že včera v městě, kde soumrak splýval\\
a ulicemi mrholil prach tesklivého deště,\\
kdes v parku nadobro setmělém kos roztouženě zpíval -- --
\end{verse}
\begin{verse}
Starého básníka knihu že opět otvírám, zdá se,\\
čtu upejpavé sloky, jež nervů nerozdrásají,\\
sentimentální nádech tkví na jejich nezralé kráse,\\
v unylou mdlobu srdce i mozek uspávají\ldots
\end{verse}
\newpage
\poemtitle{Na Skalkách}
\begin{verse}
Na Skalkách jsem stanul po bloudění\\
za ptačími hlasy v alejích.\\
Mého moře měkké rozvlnění\\
zrcadlilo nebes mladý smích.
\end{verse}
\begin{verse}
Na Skalkách dnes vítr poskakoval\\
po bělošedivých balvanech.\\
Jarníka rusáka březen koval\\
na Šumbeře nebo na Hádech\ldots
\end{verse}
\begin{verse}
Spokojeně krev mi zabušila.\\
Vítr přízniv! Děj se cokoli!\\
Dneska ráno loď má odrazila\\
pod modrými svými chocholy.
\end{verse}
\begin{verse}
Na stožárech vlajky poletují\\
jako smaragdoví motýli.\\
Nad palubou z mechu ptáci plují.\\
My se mořským dechem opili.
\end{verse}
\begin{verse}
U kormidla rošťácky se směje\\
bratr Rimbaud, chlapec divoký.\\
Na to moje české moře kleje,\\
že by musel býti bezoký.
\end{verse}
\begin{verse}
Ale když už vypluli jsme jednou,\\
tak se také spolu shodneme.\\
Nebesa než podvečerně zblednou,\\
na literaturu klejeme.
\end{verse}
\newpage
\poemtitle{Jarní kvítí}
\begin{verse}
Sedmikrásy, jaterníky,\\
bílé, žluté sasanky,\\
petrklíče, koniklece,\\
plicník, blatouch, fialky,
\end{verse}
\begin{verse}
od poupat je pozoruji,\\
doufám, čekám, miluji,\\
až je toho náhle všude\\
plná stráň a plný les.
\end{verse}
\begin{verse}
Dnem a nocí silná vášeň\\
tvoří, plodí bez hluku --\\
Každý den se znovu zpíjím,\\
do sladkého šílenství,
\end{verse}
\begin{verse}
a co nejde říci slovy,\\
%FIXME nechybí čárka na konci?
chválu každé rostliny\\
stvolů, listů, květů, barev,\\
odstíněných zelení,
\end{verse}
\begin{verse}
chtěl bych vroucně malovati,\\
aby každý list a květ,\\
každá kontura a barva\\
pěly radost, lásku, jas,
\end{verse}
\begin{verse}
jak je budí v zlaté chvíli\\
našich styků důvěrných\ldots\\
Na Východě byl to kdysi\\
v sedmnáctém století
\end{verse}
\begin{verse}
bohatý a smělý mistr\\
japonského rokoka,\\
jehož jméno bystře zvoní, --\\
byl to Korin Ogata,
\end{verse}
\begin{verse}
který štětcem zpíval chválu\\
květin tak, jak chtěl bych já.\\
Ale mistr z Orientu,\\
když své trsy rozkládal
\end{verse}
\begin{verse}
s vroucností a obratností\\
vášnivého milence\\
po skládacím paraventu,\\
dal jim vždycky za podklad
\end{verse}
\begin{verse}
vážné, pyšné, drahé zlato,\\
holedbavou nádheru --
\end{verse}
\begin{verse}
Kdežto já bych maloval je,\\
ty své jarní milenky,
\end{verse}
\begin{verse}
sedmikrásy, jaterníky,\\
bílé, žluté sasanky,\\
petrklíče, koniklece,\\
podběl, plicník, fialky,
\end{verse}
\begin{verse}
na pozadí trochu prostším;\\
hedvábí své pokryl bych\\
blankytem těch nebes našich\\
s bělostnými oblaky.
\end{verse}
\newpage
\poemtitle{V dubnu}
\begin{verse}
Oblaka bílá temné své stíny\\
pomalu vodí po vrších,\\
kde ještě mnoho hnědavé hlíny\\
a lesů sivozelených.
\end{verse}
\begin{verse}
Ale již slunce zkouší svou slávu,\\
poskakující v mracích míč,\\
své zlato pouští na zemi v trávu\\
pro pampelišky, petrklíč.
\end{verse}
\begin{verse}
Udiven leskem březové kůry,\\
koberci květů, zelení,\\
zalitý sluncem pohlížím shůry\\
na vrchů vážné vinění.
\end{verse}
\begin{verse}
Jest mi, jak bývá plavčíku v koši,\\
když na obzoru spatří břeh.\\
Volal bych: Země! Hej, hola, hoši!\\
Napravo země! V plamenech!
\end{verse}
\newpage
\poemtitle{Hučí Plaz}
\begin{verse}
Vítr duje do smrků a dubů,\\
hučí dlouhý Plaz;\\
deště míjí, stera chladných zubů\\
zatínají v sráz.
\end{verse}
\begin{verse}
Ale ptáci přece voní všude\\
s větrem o závod,\\
silná láska jejich bystře hude\\
jaru doprovod.
\end{verse}
\begin{verse}
S myslí tesknou do lesů jsem vyšel\\
cestou necestou,\\
abych větru divou píseň slyšel,\\
znavil hořkost svou.
\end{verse}
\begin{verse}
Ale ptáci nezmateni sudbou\\
jali podivně\\
srdce, jež teď čeká, až se s hudbou\\
láska vrátí v ně.
\end{verse}
\newpage
\poemtitle{Dubnové scherzo}
\begin{verse}
Na všech vrcholcích vysokých stromů\\
špačci sedí.\\
Vrzající korouhvičky dubna\\
z velmi staré mědi.
\end{verse}
\begin{verse}
Rozmrzele stříhají plachty nebe\\
šedé, neprůhledné.\\
Člověk přece před svou budkou\\
rád si na slunko sedne.
\end{verse}
\begin{verse}
Všecky vrcholky vysokých stromů\\
kolébají se s nimi.\\
Vítr rozhání s plachtami nebe\\
zbytky zimy.
\end{verse}
\begin{verse}
Hbitě na jednu usedl jsem si,\\
pluji jako v člunu.\\
Drobty slunce házím dolů\\
stromům na korunu.
\end{verse}
\begin{verse}
Stromům na korunu, na střechy dědině,\\
špačkům do hrdla rovnou.\\
Máme všichni nakonec radost\\
nevýslovnou.
\end{verse}
\newpage
\poemtitle{Stesk na jaře}
\begin{verse}
Všichni ptáci dnes mne pozdravili,\\
sojka křičela a datel zaklepal,\\
modré oči jaterníků pily\\
lačně úsměv dne, jímž lehký větřík vál.
\end{verse}
\begin{verse}
Žluťásek a babočka se chvěli,\\
mouchy sedaly na místa slunečná.\\
Šel jsem vesel, pohled můj byl smělý,\\
snad i bída má se zdála konečná\...
\end{verse}
\begin{verse}
S chladem večerním však zhasly náhle\\
bílé plaménky mé senzibility.\\
Smutku zalily je viny táhlé,\\
šerem celý svět se zdál být zalitý.
\end{verse}
\begin{verse}
Opět chuďas byl jsem živořící,\\
jenž se ničemu vzdát nesmí docela.\\
Šerem kráčel fantóm bledolící,\\
trhal poupata, jež chtivě pučela.
\end{verse}
\end{document}
