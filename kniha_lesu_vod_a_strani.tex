\documentclass{book}
\usepackage[hyphens]{url}
\usepackage{verse}
\usepackage{romannum}
\usepackage[czech]{babel}
\usepackage[utf8]{inputenc}
\title{Kniha lesů, vod a strání}
\author{Stanislav Kostka Neumann}
\renewcommand{\poemtitlefont}{\raggedright\normalfont\large\bfseries\hspace{\leftmargin}}
\begin{document}
\pagenumbering{arabic}
\maketitle
\newpage
Proč vůbec vznikla tato kniha? \textit{Kniha lesů, vod a~strání} již byla vydaná mnohokrát, koneckonců vznikla již v r.~1914. V~srpnu 2017 jsem se rozhodl, že si udělám procházku po krajinách autora Lišky Bystroušky, Rudolfa Těsnohlídka, kterou jsem nedávno dočetl a inspirovala mě tím, v~lesích u~Bílovic nad Svitavou. Procházka lesy to byla zajímavá, příroda podmanivá\ldots Až během procházky jsem z~informačních cedulí zjistil, že tuto krajinu si oblíbil i~jistý Stanislav Kostka Neumann, a~oslavoval ji ve své sbírce, kterou právě čtete.

\medskip
Po návratu domů jsem se rozhodl, že si knihu najdu a~přečtu. Stáhnul jsem z~Internetu první dokument, který jsem našel\footnote{Není to nic ilegálního, S.\,K.\,Neumann zemřel 28.\,6.\,1947 a~právě po těchto 70 letech od úmrtí autora se na jeho dílo nevztahuje autorský zákon.}, a~nelíbilo se mi, jak text vypadá. Přece jen, myslím si, že básně musí být radost nejen číst, ale i~vidět. A~hned na mě vyskočily vzpomínky na studia, kdy jsem psal práce v \LaTeX{}u. Rozhodl jsem se, že to zkusím zkombinovat: oprášit \LaTeX, přepsat v~něm Neumannovu sbírku, a~zároveň si u~toho básně důkladně pročíst.

\medskip
No a~tady ji máte. Amatérské vydání/nevydání. S~otevřeným zdrojovým textem v LaTeXu (pokud se k~vám dostalo jen PDF, tak na \url{https://github.com/michalspondr/kniha_lesu_vod_a_strani} snad najdete i~\LaTeX{}ový kód), který si můžete stáhnout či upravit k~obrazu svému. Pokud najdete překlepy či gramatické chyby, dejte mi vědět. Při přepisování jsem občas na něco podezřelého narazil, ale vzhledem k~básnické povaze knihy jsem si nebyl mnohdy jist, zda je to záměr autora nebo chyba vzniklá při jiném přepisu; ve zdrojovém kódu je možné taková místa najít v~komentáři.

\bigskip
Přeji příjemnou četbu,
\begin{flushright}
Michal Špondr\\
přepisovatel
\end{flushright}
\newpage
\tableofcontents
\newpage
% nechci zobrazovat Obsah na každé stránce
\markboth{}{}
\poemtitle{Vstupní modlitba}
\begin{verse}
Ve jménu života i radosti i krásy.
\end{verse}
\begin{verse}
Hle, země naše, ty, jež ležíš pod nebesy\\
jak žena kvetoucí pod zrádným závojem,\\
buď svato jméno tvé všem lidem po vše časy,\\
přijď nám tvé království se všemi svými plesy,\\
nás ponoř v příval svůj a zajmi sladkým snem.
\end{verse}
\begin{verse}
Buď vůle tvá nám vším, jak ptáku je a hmyzu,\\
pokorně bylině i zpívající vodě,\\
jež z drobných pramenů chce míti veleproud;\\
tvá vůle prostup nás jak uhel žíla kyzu,\\
abychom žili s ní ve světlé, moudré shodě\\
a s jasnou hrdostí tvým rodem chtěli slout.
\end{verse}
\begin{verse}
Vezdejší chléb svůj si již dobudeme sami,\\
když máme času dost na paláce a básně,\\
na lesklé sítě drah, sny, věže, kabely;\\
však síly třeba nám, jež zrušila by klamy\\
a hlucha k skuhrání klad žití zdvihla jasně\\
i naše synovství nad zápor zbabělý.
\end{verse}
\begin{verse}
A viny odpusť nám, jež nevědomosti plodí,\\
jichž dračí semeno do prostných srdcí sejí\\
sluhové fantómů a blasfemických věr.\\
Jsme děti svedené; jež bludičky nás vodí\\
do bahen ohavných, že ve své beznaději\\
ti, matko, kynem pak pro jih i pro sever.
\end{verse}
\begin{verse}
Však do pokušení nás uveď v každé chvíli,\\
vše chceme okusit, čím tělo tvé nám kyne,\\
kypící, milostné a širé tělo tvé!\\
My žádostivost svou z tvých mocných ňader pili,\\
tvá míza v poskoku se cévami nám řine\\
a lačných útrob tvých jsme květy žíznivé.
\end{verse}
\begin{verse}
Jen silné učiň nás ve víře, v lásce k tobě,\\
a jak hvozd na jaře se obrodí náš rod;\\
v temnosvit života se pohrouží jak robě\\
pro sladkou zralostí již pukající plod.\\
Tak zlého zbavíš nás jak černé snětí klasy\ldots
\end{verse}
\begin{verse}
Ve jménu života i radosti a krásy.
\end{verse}
\newpage
\poemtitle{Prolog}
\begin{verse}
Nesmrtelnou hymnu slíbil jsem kdys lesům.\\
Snadno přísahy se činí v líbánkách.\\
Slíbil jsem ji stromům, zvěři, hmyzu, vřesům.\\
Milovali jsme se v rozkošnických snách.
\end{verse}
\begin{verse}
Nesplnil jsem slibu, nesplnil ho ani,\\
když ta naše láska v štěstí uzrála.\\
Já jen o tom našem věrném milování\\
zpívám prosté rytmy lesů vazala.
\end{verse}
\begin{verse}
Také v jejich stínu střídá se sled chvíli;\\
tisíc bylo žalů, tisíc radostí:\\
Žaly, moje žaly hymnu pohltily;\\
mír a radost rády píseň pohostí.
\end{verse}
\begin{verse}
Z radostí a míru vzcházejí mé sloky,\\
smutků překonaných bývá na nich pel\ldots\\
Unikl jsem z města mílovými kroky,\\
k zdroji věčné něhy blah jsem odešel.
\end{verse}
\newpage
\poemtitle{Ocúny}
\begin{verse}
Ocúny na lukách, žluté skvrny v lesích,\\
má novou svěžest poslední tráva.\\
Umřeme v barvách, umřeme v plamenech.\\
Pak zatopí nás bělostná láva.
\end{verse}
\begin{verse}
S večerem mlhy sklouzají po lesích;\\
jak zakletý zámek tichá je řeka.\\
Mír mimo dobro a zlo se klade\\
na křídla ducha zavřená a měkká.
\end{verse}
\begin{verse}
Kdes v dáli šumí červnové vášně\\
ozvěna sladká, nesmírně táhlá.\\
Do rosy klidně složila hlavu\\
květina sluncem srpnovým zpráhlá.
\end{verse}
\begin{verse}
Ocúny na lukách, žluté skvrny v lesích.\\
Umřeme v barvách na zcela malou chvíli.\\
Básníci života, jsme jako země:\\
zas rozkvetou růže, jež jsme zasadili.
\end{verse}
\begin{verse}
Na jednu zimu zapomenuti,\\
k novému jaru z mrtvých povstaneme.\\
Ocúny na lukách, žluté skvrny v lesích:\\
od ocúnů k prvním sněženkám jdeme.
\end{verse}
\newpage
\poemtitle{Září}
\begin{verse}
Ty, sladký září! Jak je modré nebe\\
nad těmi vrchy, kde se barvy rodí!\\
Jdu zmaten vzduchem, jenž je plný tebe,\\
a chtěl bych plouti povětrnou lodí.
\end{verse}
\begin{verse}
Tak zcela nízko nad lesy, jež mění\\
svou píseň zelení na píseň žlutí,\\
a v slunci tiše oddati se snění\\
o novém jaru, novém zahynutí.
\end{verse}
\begin{verse}
I tak je dobře však na zemi něhy,\\
jež dovede tak krásně umírati\\
na chvíli, požár uhašený sněhy,\\
by znovu vzplála s jarem, jež se vrátí.
\end{verse}
\begin{verse}
Již v lesích strání měď a zlato hoří,\\
koňadra u cest nepokojně hvízdá\\
v klid, jenž se snáší ze sivého boří\\
a v němž svou radost poslední teď hmyz dá.
\end{verse}
\begin{verse}
Motýla zdvíhám létem znaveného\\
a zas ho pouštím na poslední květy;\\
ještě se těšit budem ze dne svého\\
já, cikády a ptáci, zvěř a květy\,\ldots
\end{verse}
\begin{verse}
Proč prchají však světle hnědé srny?\\
Můj krok přec tišší nemůže už býti.\\
Jsem tulák trochu rozedraný trny\\
a teplo zvířat chtěl bych pocítiti!
\end{verse}
\newpage
\poemtitle{Oběť poděkovací}
\begin{verse}
Těch ohňů podzimních a jejich namodralých kouřů,\\
jež rozptylují se tak příliš pomalu!\\
Jsou jako modlitby či oběť poděkovací\\
za všecky rozkoše letního zápalu.
\end{verse}
\begin{verse}
Za mladá těla žen a jejich supějící vášeň,\\
za slunce skvělý dar od června do září,\\
za lesů vonný stín a louky rozezpívané,\\
za květy, ptáky, vody, hudbu komáří\,\ldots
\end{verse}
\begin{verse}
A mráčky kouře děkují\ldots A nevědouce komu\\
do lesů tiše plynou, kde již usíná\\
pod rzí a krví stromů skála znovu tvrdnoucí,\\
jež slední vzdech svých ňader pevně upíná\ldots
\end{verse}
\begin{verse}
Poutníče, zvěstuj v nížinách a v sazovitých městech,\\
my že tu šťastni byli s lesy po horách,\\
že také děkujeme slunci\ldots zemi chladnoucí\ldots\\
za život pohanský, sen v barvách, lesklý prach\,\ldots
\end{verse}
\newpage
\poemtitle{Dubisko padlo}
\begin{verse}
Za mostem u Myší díry jednoho rána v září\\
veliké dubisko padlo z úpatí skalnaté stráně,\\
na louku zrosenou padlo k slunci tíhnoucí tváří;\\
dělníků osm tu stálo, listí pršelo na ně.
\end{verse}
\begin{verse}
Sekyrou podťato hlučně a pilou podřezáno\\
do trávy zrosené padlo, až v kučeravé hlavě\\
tisíce stonů zavzdychlo a zapraštělo v ráno,\\
jež rez a žluť a červeň strání odhalovalo právě.
\end{verse}
\begin{verse}
Svitava vzdorně hučela, do balvanů bila,\\
strhujíc v bělostnou pěnu zlaté habrové listí,\\
nad černavými bory kavka se rozkroužila:\\
dělníků osm tu stálo, lidé, jimž se chce jísti.
\end{verse}
\begin{verse}
Dělníků osm tu stálo z rozkazu knížete pána,\\
nepřátel osm tvrdých vtrhlo sem z blízké vísky,\\
v samotu, v ticho za řekou: v šest padla první rána;\\
za dubem padnou duby, habry, břízy a lísky.
\end{verse}
\begin{verse}
Na sta jich padne, útlé i okoralé již kmeny,\\
v bok stráně vnikne kopáč, až jiskry vydá kámen,\\
balvany zřítí se v řeky tok hlučný a rozpěněný,\\
mech v hlíně zalkne se udupán a hraboš prchne zmámen.
\end{verse}
\begin{verse}
To cestu rozkázal budovat náš katolický pan kníže;\\
do Dlouhých strání se povine vzhůru tišinou lesů,\\
by z rodné hlíny v údolí a k nádraží měly blíže\\
mrtvoly smrků a sosen padlých v bitevním děsu.
\end{verse}
\newpage
\poemtitle{Tam, kde město počíná}
\begin{verse}
Ze starých topolů tu prší srdce sivá,\\
do řeky, zkažené mdlou špínou předměstí,\\
v níž továrenský kal a barva nepravdivá\\
zpěv zkouší duhový o shnilé neřesti.
\end{verse}
\begin{verse}
Tok teplých odvarů a splašek ustálených\\
lesklými žilami mdlá žíhá mastnota;\\
ryb těla stříbrná od břehů prchla zděných\\
plesnivým kamenem, v němž bahno klokotá.
\end{verse}
\begin{verse}
Jak zpustlý hřbitov tu i dno je zneuctěno,\\
kde mrtvá koťata se jistě válejí;\\
puch stoupá soumrakem, a vše je zachmuřeno,\\
když tudy otroci jdou domů z galejí.
\end{verse}
\begin{verse}
A umírání sil je tady teprv smutné,\\
tak jako renouveau je k pláči žalné tu;\\
jen zima přijde sem jak smilování nutné\\
a hanbu ukryje v svém bílém sametu
\end{verse}
\begin{verse}
snad aspoň na dnů pár, ač neuvěříš ani,\\
že také tenhle kout se někdy rozjasní\ldots\\
Och, jak je krásné teď tam u nás umírání,\\
kde vrchy planoucí o zašlém létu sní
\end{verse}
\begin{verse}
a všechen slunce jas, jejž dlouhým douškem pily,\\
teď v nach a ve zlato svých lesů vdechují -- \\
och, jak je svěží teď a jasně rozpustilý\\
tam u nás řeky proud, s nímž vlny strhují
\end{verse}
\begin{verse}
od rodných břehů vrb a olší listí sváté\ldots\\
Zde je vše prokleto a zhanobeno však,\\
vzduch, půda, řeka, strom i dětí tělo zlaté\\
i oprýskaná zeď, jež ční jak ztvrdlý mrak.
\end{verse}
\begin{verse}
A místo pryskyřic a lesů bílé páry\\
již z dlouhých komínů dým stoupá k blankytu:\\
před sluncem ukrývá své zmrzačené tvary\\
kout země zchátralé a zmírající tu.
\end{verse}
\newpage
\poemtitle{Na podzimním slunci}
\begin{verse}
Obelstěn sluncem, vzešlým opožděně\\
nad dohořívajícím požárem,\\
jejž podzim zažehl, jak zbité štěně\\
nějaké rány léčit krásným dnem\\
se vleku na výsluní mezi lesy\\
a k pařezu dím: Odpočineme si.
\end{verse}
\begin{verse}
Sám věru nevím, oč tu vlastně běží,\\
proč duch i tělo zmalátněly tak,\\
proč bolest v nitru nevrle se ježí\\
a smutek plíží se jak polem svlak,\\
jenž stonky obepíná, dusí květy\\
a jehož zbytky vždy jsou nevyplety.
\end{verse}
\begin{verse}
Sám věru nevím, kterou zlatým jitrem\\
teď léčím ránu. Stero proniká\\
střel otrávených otupělým nitrem.\\
Jeť žalostný dnes osud básníka\\
ve vlastech, kterým k prostitutek smíchu\\
jsou mladá srdce zpívající v tichu.
\end{verse}
\begin{verse}
Snad muž, snad žena, dav snad, země celá,\\
och, což já vím, kdo nejvíc poranil!\\
Ze stráně, která včera dohořela,\\
se třesu běláskem, jenž v říjen zbyl,\\
a klamným sluncem dávám obelstíti\\
den jeden svůj pro anemické kvítí.
\end{verse}
\newpage
\poemtitle{Pytláci}
\begin{verse}
Říjnové ráno dozrává již v mlze, tichu a chladu\\
nad řekou hustou a olověnou, jež zase trochu klesla,\\
kolkolem jak by se rozptylovala voda v dychtivém hladu,\\
na vodě naše pytlácká loďka: šplouchají červená vesla.
\end{verse}
\begin{verse}
Napravo, nalevo po stěnách strání sotva se tuší lesy,\\
jež zcela jistě z mlhy dnes vyjdou holejší zas a teskné;\\
v šedivém vlhku váhavě plujeme hluboko na dně kdesi,\\
vetřelci, na něž se ryba jde dívat a na hladinu pleskne.
\end{verse}
\begin{verse}
Šplouchají vesla, mlha se válí, vrána pozdravuje\\
den plížící se do zšedlých borů i do hnědých dubin.\\
Po vodě naše pytlácká loďka jako ve snách pluje,\\
z vrch sivých prchají vodní žínky do neprůzračných hlubin\ldots
\end{verse}
\begin{verse}
A již se cos hnulo. Stoupají mlhy po vrších z vodního klínu\\
do lesů nahoru, oblaka šedá mezi nebem a námi,\\
z vrcholu stráně k vrcholu naproti klenutí světlých stínů,\\
tragická vteřina před bitvou se sluncem, jež vstalo za horami.
\end{verse}
\begin{verse}
Na vodě naše pytlácká loďka\,\ldots \uv{Chval každý duch Hospodina!}\\
Blýsklo se za námi nahoře na vrchu! Z borů se vyhoupl náhle\\
stříbrný kotouč tajemně svítící v mlhy oblaka siná,\\
a v dáli jako by zazněly polnic fanfáry táhlé.
\end{verse}
\begin{verse}
% FIXME "položme" velkým?
Hej, holá! položme červená vesla, na kořist nemysleme\\
a sepněme ruce v modlitbě tiché k tomu, jenž líbá zemi\\
a jemuž tak skoro za vše, co máme, za život děkujeme\\
\ldots i za tu rybu, jež se třepe snad chycená pod olšemi.
\end{verse}
\newpage
\poemtitle{Se složenými vesly}
\begin{verse}
\textit{\ldots es durchweht mit ein Erkennen,\\
wie grenzenlose Weiten Meschnen trennen,\\
wie furchtbar einsam unsre Seelen leben\ldots\\
H. von Hofmannsthal}
\end{verse}
\begin{verse}
Složil jsem vesla. Po vodě loďka kolébá se a plyne.\\
Je řeka tu klidná jak hluboký rybník mezi blízkými břehy,\\
kde vrbové pruty k olšové větvi chmelová liána vine,\\
kde z šedého stříbra vrbových houštin list padá jak slza něhy.\\
Složil jsem vesla. V soumraku, v mlhách podzimní píseň hyne.
\end{verse}
\begin{verse}
Hladové noci nám upíjejí malátný půvab denní,\\
jenž velké své štěstí v slunečním jasu bez tepla poznává stěží,\\
po lesích strání plameny hasnou v zimomřivém chvění:\\
bůh ví, kde ptáci jsou, kde květin všechna semena leží.\\
Složil jsem vesla. V soumraku, v šedi život je zapomnění.
\end{verse}
\begin{verse}
V mokvavém chladu na loďce sedím\ldots jako by nebylo lidí\\
za těmi mlhami, za těmi lesy, jako bych tady byl doma\\
pod olšemi k vodě skloněnými, kde parma vousatá slídí,\\
srn, jež jdou pít sem, ryb a kavek bratr s vyschlýma rtoma.\\
Myslím si: Zasil jsem, ať kdo chce, co kde chce a pro kohokoliv sklidí.
\end{verse}
\begin{verse}
Od divokých břehů šílených měst přišel jsem k vodě a lesům,\\
já, který se učím teď chápati jich nedůvěřivost k lidem,\\
já, který se naučil rozuměti jejich nejmenším hlesům\\
a daleko lidí býti šťasten i trpěti s jejich klidem;\\
svých včerejšků neznám a svoje zítra zpívám na vrších vřesům.
\end{verse}
\begin{verse}
Od člověka k člověku nesmírná cesta a ještě nedojdeš k cíli;\\
se stromy a balvany, s travou a řekou, se zvěří, s hmyzem jsem jedno.\\
Chci, aby bory, louky a vody voněly z dnů mých a chvílí:\\
jsem zde teď, haluzka domácí půdy, hrouda a já jsme jedno,\\
pro sebe jsme se narodili, abychom sobě žili\ldots
\end{verse}
\begin{verse}
Složil jsem vesla. Stíny se stýkají. Jak je všechno krásné!\\
I pomalé zmírání znavených krajů pro zimní klid a spánek,\\
jenž přichází se slibem širých ploch, jež bělostné jsou a jasné.\\
Loďka se kolébá. Hladinu čeří po proudu studený vánek.\\
Zde šťastný život pohádkou není, zde se šťastně i hasne.
\end{verse}
\newpage
\poemtitle{Listopad mezi buky}
\begin{verse}
Všechny jízvy strání se již obnažily\\
zvětralé a šedé, mechem skvrnité.\\
Suchá voda větru šumí steskem lesů,\\
po vrších je slunce měkce rozlité.
\end{verse}
\begin{verse}
Slunce milosrdné, jež by chtělo hřáti\\
stromů vrcholky již polobezlisté\ldots\\
chladný den má vůni hrdé rezignace,\\
barvy rozkladu a lesky zlatisté.
\end{verse}
\begin{verse}
Bloudím žleby mezi bukovými lesy.\\
Ještě mají stromy trochu zeleně,\\
vydechující do spleti kmenů choře\\
lehce fialovou páru jeseně.
\end{verse}
\begin{verse}
Bloudím tiše s křídly klidně složenými,\\
zhnědlé zlato buků šustí pod nohou,\\
na dně žlebu ručej zvoní ledovitě,\\
ptáci -- touho! -- ptáci zpívat nemohou\ldots
\end{verse}
\begin{verse}
Stojím v klíně kopců; harmonií stesků\\
sladce dýše píseň listopadová.\\
A mne mrzí jedno: v černém haveloku\\
že tu stojím jako skvrna surová.
\end{verse}
\begin{verse}
Raděj byl bych faunem chlupatým a hnědým,\\
v hnědých proudech listí sotva zřejmý bod:\\
na bukový pařez s píšťalou bych used\\
listům padajícím hráti doprovod.
\end{verse}
\newpage
\poemtitle{Doma}
\begin{verse}
Mlhy jsou v kotlinách a bez lesků sivé je nebe;\\
vejde-li člověk do lesů, hlavu zadumán svěsí.\\
Po rusých temenech vrchů zavřených v sebe\\
na šedých šlářích ticho zkřehlými pluje lesy.
\end{verse}
\begin{verse}
Datel jen odklepává hubený den svůj a těká\\
po sosnách fialových. Sýkora v habří pískne.\\
Melancholická radost a barev dřímota měkká\\
pod nebem uzavřeným k zemi přátelsky tiskne.
\end{verse}
\begin{verse}
V teplém tak hnízdě ptáku dobře a veselo bývá,\\
hledí-li mřežovím větví z rodného stromu\ldots\\
Obloha má ovšem slunce, ale je nepravdivá.\\
K zemi se navracíš. I když je chudá. Domů.
\end{verse}
\newpage
\poemtitle{Vločky jdou}
\begin{verse}
Zatáhla se nebesa těžkou šedí sněhovou,\\
milióny vloček jdou, rozsypou se: hou, hou, hou,\\
se zemí se nebesa spojí sítí ledovou,\\
země ztratí špinavá unavenou barvu svou.
\end{verse}
\begin{verse}
Radosti své rozloží bílé plochy veselé,\\
neubrání se jí ni černé bory na horách,\\
a jen řeka kalná, mdlá jako z tuhy vyvřelé,\\
zhltne vločku za vločkou, světélka, jež mizí v tmách.
\end{verse}
\begin{verse}
Spadne k zemi velký mír s šumem skoro neslyšným,\\
\uv{odpusťme si, co jsme si,} šeptly vrchy, pole, les;\\
bílé ticho ulehne se životem bezdyšným\\
v širý kraj a širý klid vanoucí až do nebes.
\end{verse}
\begin{verse}
Krtek spí již v doupěti, v hroudě strnul drobný hmyz,\\
zrno jasně vzklíčené čeká povlak sněhový,\\
holé stromy svírají zastavené proudy míz,\\
skála o svých nadějích ničeho již nepoví\ldots
\end{verse}
\begin{verse}
Přes noc spadne bílý div do polí a do lesů,\\
bílý mír a bílý jas, pokoj všemu stvoření.\\
K zásypům svým půjde zvěř, v oku klid a bez hlesu,\\
v hluboký a tichý žleb, kde se vzpomíná a sní.
\end{verse}
\begin{verse}
Sní a čeká\,\ldots Člověk jen v děrách svých v ulicích\\
hladem, krví, mozkem štván šílenství svých nezmění;\\
v bílý den i v bílou noc řičí jeho řev i smích,\\
neustálý boj o život, neustálé říjení.
\end{verse}
\newpage
\poemtitle{Vysoko uprostřed lesů}
\begin{verse}
Vysoko uprostřed lesů stojím o šedé dubisko opřen,\\
přišel jsem z bláta dědiny sem, kde třpytí se bílý sníh,\\
kde křesťanů bůh je neznám zcela a každou haluzí popřen,\\
kde nad čerstvou stopou zvěře táhne dech míru po vrších.
\end{verse}
\begin{verse}
Napravo, nalevo, dokolo kolem mřežovím kmenů a větví\\
na ztemnělé moře vrchů vidím, kde u lesů stojí les;\\
jeho rozloze klidné a hrdé jen dravého ptáka let ví,\\
o jeho hloubce liška snad zaštěká z mlh ranních do nebes.
\end{verse}
\begin{verse}
U vlny černozelené vlna, která je černohnědá,\\
ční nepohnutě a mlčky z moře jak v zakletí a snách;\\
nad nimi visí se slibem sněhů obloha bez konce šedá:\\
uprostřed malý-veliký stojím tu: květ kvete mi na vlnách.
\end{verse}
\begin{verse}
Ze spící skály a nad bílé sněhy pomněnkově kvete,\\
nahoře v lesích jej opatruji a vlastní mou krví je živ,\\
substance krve mé dává mu lesky, tvar jeho měkce hněte;\\
země a člověk krvesmilnili a zplodili modrý div.\\
\end{verse}
\begin{verse}
Zplodili o jedno štěstí více, upřímné, veselé, prosté,\\
jak zrno z hroudy vyňaté sluncem, jež směje se na líchu;\\
v září mi uzrálo uprostřed plamenů, nad sněhy teď mi roste,\\
vod, hlíny, medů a pryskyřic má vůně v kalichu\,\ldots
\end{verse}
\newpage
\poemtitle{Leden}
\begin{verse}
Nad sněhovými plochami spí černých koster lesy\\
po stráních, které zbělely a nesou nahé jizvy\\
ve svit, jenž stoupá od země a v šeď ztlel pod nebesy.\\
%FIXME Tu poutník - oprava chybějící čárky
Tu poutník, když se objeví, jak ruka zpupné výzvy
\end{verse}
\begin{verse}
v tvář ticha, spánku, osudu ční nad studené sněhy.\\
%FIXME čárka za smrti
Den krutý je a přísnější než pocel samé smrti\\
%FIXME v jiném zdroji je na začátku "--", snad místo chybějící čárky před "smrti" nad tím?
-- jež v lůně nových životů vždy slib má plný něhy --\\
den krutý je a každý hlas bílými spáry škrtí.
\end{verse}
\begin{verse}
A mezi ledy břehů svých tok řeky zeje mrtvý,\\
tajemnou hloubkou hedvábnou v chlad svého ticha láká\\
jak propast táhlá, řekl bys, o které tu jen smrt ví.\\
Však černá voda zrcadlí vrb těla křivolaká.
\end{verse}
\newpage
\poemtitle{Zimní noc}
\begin{verse}
To není země, to je sen,\\
jejž bledý měsíc vykouzlil\\
pod zastíněnou oblohou\\
na vlnách sametových mil.
\end{verse}
\begin{verse}
To není země, to je div,\\
nesmírná hudba bílých cest,\\
jež pro své tiché jiskření\\
do hlubin strhly světlo hvězd.
\end{verse}
\begin{verse}
To není země, to je zjev,\\
jenž vyplul z nekonečnosti.\\
Já, blázen, bod a cizinec,\\
jdu směšný mraznou věčností.
\end{verse}
\newpage
\poemtitle{Prosté sloky}
\begin{verse}
\large{\Romannum{1}}
\end{verse}
\begin{verse}
Miluji hvězdná nebesa\\
pro jejich hloubku a krásu,\\
pro jejich modrou záhadu\\
plničkou třpytného jasu.
\end{verse}
\begin{verse}
Napohled úsměv, ticho, mír,\\
ve skutečnosti strž světů\\
nejhezčí, nejpodivnější\\
bez bohů a jejich tretů.
\end{verse}
\begin{verse}
Stanul jsem v noci lednové\\
uprostřed nesmírných sněhů:\\
velebnost hmoty chápal jsem\\
a v srdci pocítil něhu.
\end{verse}
\begin{verse}
\large{\Romannum{2}}
\end{verse}
\begin{verse}
Nevíme kam a nevíme proč,\\
záhady všude je tolik;\\
nebesa hvězdná podivná jsou,\\
podivný hvězdnatý dolík,
\end{verse}
\begin{verse}
tak jako každý života děj,\\
černých těch borů tam zrání --\\
i to, že dřepič z kaváren, já,\\
šťasten jsem na sněžné pláni.
\end{verse}
\newpage
\poemtitle{Ojíněly lesy}
\begin{verse}
Ojíněly husté smrky, dlouhé borovice,\\
zkřehle věsí jehličí své, jak pod bílou plísní;\\
sněžné skvrny příliš hutně imitují slunce,\\
sýkor pískot nesmělý chce býti možná písní.
\end{verse}
\begin{verse}
Ojíněly kostry habrů, modřínů a dubů,\\
na mlázi jen hnědé listí schlíple ještě visí;\\
velké ticho dříme v mlze měkké, šeré, vlhké,\\
která s jemným steskem věcí důvěrně se mísí.
\end{verse}
\begin{verse}
Ojíněla stébla trávy v sněžnou vegetaci,\\
jež se láme pod kročeji, neduživě praská,\\
velcí ptáci vzletí tiše, mizí v srdce lesů:\\
samota mě svými prsty hedvábnými laská.
\end{verse}
\begin{verse}
Je v tom trochu život a trochu vlídné smrti,\\
odhozené iluze a nadějí svět celý.\\
Sentimentální by byla žena tudy jdoucí,\\
ale básník slyší všude spodní akord vřelý:
\end{verse}
\begin{verse}
%FIXME silu bylo v jednom zdroji
důvěru a klidnou sílu v životi i smrti,\\
pod melancholickou tváří v krásný osud víru\ldots\\
V chladu stesků ojínělo také jeho srdce,\\
ale v hlubině zní píseň pohody a míru.
\end{verse}
\newpage
\poemtitle{Spánek únorový}
\begin{verse}
Mlhavý den únorový, hnědá převládá.\\
Poprašek, jenž v noci spadl, tvoří trochu skvrn,\\
bílé kontury a stezky. Pustá nálada\\
vane nad stopami kavek, nad stopami srn.
\end{verse}
\begin{verse}
Lesy jako vyhořelé dechu nemají,\\
mrtvá hnízda obnažená rozpadají se,\\
všechen život jako když se v pupen utají\\
zakletý a spící pevně v tvrdém obryse.
\end{verse}
\begin{verse}
Ztuhlou zemi zšedlá tráva měkčí nečiní,\\
ani mechy vysílené, listí vyrudlé;\\
zvěř a pták jsou pouhé stíny v šeré jeskyni,\\
jež se plíží tichem vrchů jako po truhle -- -- --
\end{verse}
\begin{verse}
Nenoste sem rudých ohňů velkých pochodní!\\
Nenoste sem vřesku polnic, větrů šílení!\\
Je to maska, spadne náhle, světy zavodní\\
pod úsměvem modrých nebes proudy zelení.
\end{verse}
\newpage
\poemtitle{Noc přípravy}
\begin{verse}
Srdce, slyšíš? Duní to a ječí\\
v temnou noc, jež mlhou napojena.\\
Řeka vzepjala se náhlou křečí,\\
k porodu se chystající žena,\\
slavné dění počlo za horami,\\
kde se tká již úsměv ze zlata,\\
z noci severní vstříc jde mu tmami\\
puklých ledů kantáta.\\
Srdce, slyšíš? Buší na tvé stěny.\\
%FIXME Otevři bylo původně
%FIXME V jiném zdroji byla za "Čas je" čárka a proto malé písmeno u "otevři se"
Čas je. Otevři se, než den vzplá.\\
Přede dveře postav na stráž feny,\\
aby člověčina zpozdilá\\
nezasedla trůnů, které patří\\
bohům lesů, žlebů, luk a vod:\\
Až mé oči první zeleň spatří,\\
už ti neublíží svod!
\end{verse}
\newpage
\poemtitle{Březen}
\begin{verse}
Ještě chvilku jen! A moje vděčnost\\
opět pozdraví tvoje milosrdenství.\\
Lidé ciframi měří tvou užitečnost,\\
čím však mně jsi, mé srdce ví.
\end{verse}
\begin{verse}
Země, zítra se zazelenáš!\\
Země, zítra nám rozkveteš!\\
Nikdo nepodplatí tě, nekoupí za otčenáš,\\
ale věrnému synu vším se zveš.
\end{verse}
\begin{verse}
Zpěvem, letem a hnízděním ptáků,\\
vůní, ztepilostí a barvami bylin svých,\\
víc než zlatem klasů mně rudostí máků\\
tisíc radostí učiníš nezasloužených.
\end{verse}
\begin{verse}
Šerou pohádkou vody, luk šťavnatostí,\\
lesů ševelem, zvěří a pryskyřicemi\\
osvobodíš mě z bídy, učitelko ctností,\\
dechem života naplníš plíce mi.
\end{verse}
\begin{verse}
%FIXME Nemá být Verunka?
Nedočkavá verunka v sednici hledá kvítí.\\
Spolu toužíme, březnovou nemocí stůněme.\\
Lidem nemajíce zač vděčni býti,\\
celou vděčnost svou, země, vstříc ti neseme.
\end{verse}
\newpage
\poemtitle{U Děravé skály}
\begin{verse}
Na šedém balvanu Děravé skály\\
se samotou srůstaje sedím,\\
pode mnou vody jak by se rvaly,\\
na zběsilou Svitavu hledím.
\end{verse}
\begin{verse}
Rozbouřená písnička vody,\\
mé mladosti, mých nadějí parafráze,\\
podjarní předzvěst nové budoucí shody\ldots\\
jí naslouchám a je mi blaze.
\end{verse}
\begin{verse}
Pleskají, hučí, šumí a zvoní vody,\\
žlutavé vlny letí, rozbíjí se a pění,\\
vzduch revolučními chvěje se svody;\\
však lesy strání dumají v snění.
\end{verse}
\begin{verse}
Jakož jim kážou zákony rodné hlíny,\\
oddaně mízy vzestup očekávají,\\
své věrny zemi, věrny sobě pod oblačnými stíny\\
pevně se drží a zrají.
\end{verse}
\begin{verse}
I já tu přitisknut ke skále chladné\\
své půdě i sobě věrnost přísahám celou,\\
bez lítosti nad tím, co ve mně vadne,\\
tomu, co klíčí ve mně, oddanost slibuji vřelou.
\end{verse}
\begin{verse}
Svoboden býti, toť kvésti a zráti\\
po zákonech svého růstu,\\
neohlížet se napravo, vlevo nešilhati,\\
nevěřit hýření, neřkuli půstu.
\end{verse}
\begin{verse}
Na věky hotov s bohem, s osudem usmířený,\\
marnými blasfémiemi nedráždím nervů;\\
miluji slunce, zemi, lesy, vody a ženy --\\
s lidmi se, když je to nutné, bez bázně servu.
\end{verse}
\begin{verse}
Jen abych nezradil sebe, toho jest nejvíce dbáti,\\
bytosti rytmus sblížiti s rytmem země,\\
na pevné půdě co nejpevněji státi\\
a vteřinu pochopit jemně\ldots
\end{verse}
\newpage
\poemtitle{Jarní zvěstování}
\begin{verse}
Tak dlouho čekali jsme rozechvěni touhou\\
a chladem, který vál ze sněhů svítících,\\
den žití chtěli jsme dát za sněženku pouhou,\\
však blesky chladnými jen vysmál se nám sníh.
\end{verse}
\begin{verse}
Až náhle začlo tát a vzedmula se řeka,\\
divadlo veliké pod lesy hrály kry;\\
my zřeli s rozkoší, jak dole proud se vzteká,\\
a připravovali své jarní mimikry.
\end{verse}
\begin{verse}
Však březen závistně nám nové poslal sněhy,\\
sen sněženkových spoust nade vším rozestřel;\\
tou bílou pohádkou, již z hebké utkal něhy,\\
my šli jsme zmateni a v srdcích měli žel.
\end{verse}
\begin{verse}
%FIXME květ, co
A neupřímný květ co jitro sněžil venku,\\
by slunci vzdoroval, jež přišlo k poledni;\\
kůl každý v plotě sníh měl skvělou za čelenku;\\
však my jen čekali na sněhy poslední.
\end{verse}
\begin{verse}
Tak dlouho čekali jsme touhou rozechvěni.\\
Dnes\ldots atmosféra jest jak brána dokořán\\
ve svěžest, modro, jas a jitřní kuropění,\\
dnes první sladký vznět byl země lůnu dán.
\end{verse}
\begin{verse}
Zdrávas, země,\\
plná milosti,\\
požehnaná jsi mezi květy,\\
země,\\
požehnaný plod života tvého,\\
člověk!
\end{verse}
\begin{verse}
Cos méně zřejmého než nádech neurčitý\\
a jistějšího přec nad všechnu jistotu,\\
v kraj spánkem znavený se sune s mdlými svity\\
a klade první šlář na jeho nahotu.
\end{verse}
\begin{verse}
Již ptáku moudrému po milence se stýská,\\
již rouchou svatební má pěnkava i drozd;\\
ze sterých hrdélek změť tónů ryzích tryská:\\
za sedmi řekami spí chladná minulost.
\end{verse}
\begin{verse}
Však naší řeky proud již klesající zvolna\\
ke ptačím písničkám svůj zpívá doprovod,\\
jenž stoupá po stráních jak hudba sladkobolná,\\
jak snění dívenky, jež v první klesla svod.
\end{verse}
\begin{verse}
Ve skrytých poupatech se rodí cudné tvary,\\
proud barev mladistvých již čeká v pupenech;\\
by v tělech roznítil nám sladce tajné žáry,\\
na každé haluzi se jitří vonný dech.
\end{verse}
\begin{verse}
To dekor slavnostní se pro dny snubní staví,\\
pro hříčky milostné se stelou koberce\ldots\\
Jak lačné kalichy své vztyčujeme hlavy\\
a vůně omamná nám stoupá ze srdce.
\end{verse}
\begin{verse}
Zdrávas, země,\\
plná milosti,\\
požehnaná jsi mezi světy,\\
země,\\
požehnaný plod života tvého,\\
člověk!
\end{verse}
\newpage
\poemtitle{Jiné renouveau}
\begin{verse}
Rozhučela se Svitava kalná a rozvodněná,\\
že břehy nestačí nikterak špinavě hnědému proudu.\\
Své jehnědy z vody zdvíhají keřiska potopená.\\
Záchvěvy roztávání vzrušují zmrzlou hroudu.
\end{verse}
\begin{verse}
Rozhučela se Svitava kalná a rozvodněná.\\
Je ve vzduchu naděje plno a cudných paprsků března,\\
jak by mi po boku kráčela s nesmělou láskou žena,\\
jež tuší, že dnes první prchavé polibky sezná.\\
A před lesů tváří tichou a jakoby lhostejnou ještě\\
teď vzpomínám si, že včera v městě, kde soumrak splýval\\
a ulicemi mrholil prach tesklivého deště,\\
kdes v parku nadobro setmělém kos roztouženě zpíval -- --
\end{verse}
\begin{verse}
Starého básníka knihu že opět otvírám, zdá se,\\
čtu upejpavé sloky, jež nervů nerozdrásají,\\
sentimentální nádech tkví na jejich nezralé kráse,\\
v unylou mdlobu srdce i mozek uspávají\ldots
\end{verse}
\newpage
\poemtitle{Na Skalkách}
\begin{verse}
Na Skalkách jsem stanul po bloudění\\
za ptačími hlasy v alejích.\\
Mého moře měkké rozvlnění\\
zrcadlilo nebes mladý smích.
\end{verse}
\begin{verse}
Na Skalkách dnes vítr poskakoval\\
po bělošedivých balvanech.\\
Jarníka rusáka březen koval\\
na Šumbeře nebo na Hádech\ldots
\end{verse}
\begin{verse}
Spokojeně krev mi zabušila.\\
Vítr přízniv! Děj se cokoli!\\
Dneska ráno loď má odrazila\\
pod modrými svými chocholy.
\end{verse}
\begin{verse}
Na stožárech vlajky poletují\\
jako smaragdoví motýli.\\
Nad palubou z mechu ptáci plují.\\
My se mořským dechem opili.
\end{verse}
\begin{verse}
U kormidla rošťácky se směje\\
bratr Rimbaud, chlapec divoký.\\
Na to moje české moře kleje,\\
že by musel býti bezoký.
\end{verse}
\begin{verse}
Ale když už vypluli jsme jednou,\\
tak se také spolu shodneme.\\
Nebesa než podvečerně zblednou,\\
na literaturu klejeme.
\end{verse}
\newpage
\poemtitle{Jarní kvítí}
\begin{verse}
Sedmikrásy, jaterníky,\\
bílé, žluté sasanky,\\
petrklíče, koniklece,\\
plicník, blatouch, fialky,
\end{verse}
\begin{verse}
od poupat je pozoruji,\\
doufám, čekám, miluji,\\
až je toho náhle všude\\
plná stráň a plný les.
\end{verse}
\begin{verse}
Dnem a nocí silná vášeň\\
tvoří, plodí bez hluku --\\
Každý den se znovu zpíjím,\\
do sladkého šílenství,
\end{verse}
\begin{verse}
a co nejde říci slovy,\\
%FIXME nechybí čárka na konci?
chválu každé rostliny\\
stvolů, listů, květů, barev,\\
odstíněných zelení,
\end{verse}
\begin{verse}
chtěl bych vroucně malovati,\\
aby každý list a květ,\\
každá kontura a barva\\
pěly radost, lásku, jas,
\end{verse}
\begin{verse}
jak je budí v zlaté chvíli\\
našich styků důvěrných\ldots\\
Na Východě byl to kdysi\\
v sedmnáctém století
\end{verse}
\begin{verse}
bohatý a smělý mistr\\
japonského rokoka,\\
jehož jméno bystře zvoní, --\\
byl to Korin Ogata,
\end{verse}
\begin{verse}
který štětcem zpíval chválu\\
květin tak, jak chtěl bych já.\\
Ale mistr z Orientu,\\
když své trsy rozkládal
\end{verse}
\begin{verse}
s vroucností a obratností\\
vášnivého milence\\
po skládacím paraventu,\\
dal jim vždycky za podklad
\end{verse}
\begin{verse}
vážné, pyšné, drahé zlato,\\
holedbavou nádheru --
\end{verse}
\begin{verse}
Kdežto já bych maloval je,\\
ty své jarní milenky,
\end{verse}
\begin{verse}
sedmikrásy, jaterníky,\\
bílé, žluté sasanky,\\
petrklíče, koniklece,\\
podběl, plicník, fialky,
\end{verse}
\begin{verse}
na pozadí trochu prostším;\\
hedvábí své pokryl bych\\
blankytem těch nebes našich\\
s bělostnými oblaky.
\end{verse}
\newpage
\poemtitle{V dubnu}
\begin{verse}
Oblaka bílá temné své stíny\\
pomalu vodí po vrších,\\
kde ještě mnoho hnědavé hlíny\\
a lesů sivozelených.
\end{verse}
\begin{verse}
Ale již slunce zkouší svou slávu,\\
poskakující v mracích míč,\\
své zlato pouští na zemi v trávu\\
pro pampelišky, petrklíč.
\end{verse}
\begin{verse}
Udiven leskem březové kůry,\\
koberci květů, zelení,\\
zalitý sluncem pohlížím shůry\\
na vrchů vážné vinění.
\end{verse}
\begin{verse}
Jest mi, jak bývá plavčíku v koši,\\
když na obzoru spatří břeh.\\
Volal bych: Země! Hej, hola, hoši!\\
Napravo země! V plamenech!
\end{verse}
\newpage
\poemtitle{Hučí Plaz}
\begin{verse}
Vítr duje do smrků a dubů,\\
hučí dlouhý Plaz;\\
deště míjí, stera chladných zubů\\
zatínají v sráz.
\end{verse}
\begin{verse}
Ale ptáci přece voní všude\\
s větrem o závod,\\
silná láska jejich bystře hude\\
jaru doprovod.
\end{verse}
\begin{verse}
S myslí tesknou do lesů jsem vyšel\\
cestou necestou,\\
abych větru divou píseň slyšel,\\
znavil hořkost svou.
\end{verse}
\begin{verse}
Ale ptáci nezmateni sudbou\\
jali podivně\\
srdce, jež teď čeká, až se s hudbou\\
láska vrátí v ně.
\end{verse}
\newpage
\poemtitle{Dubnové scherzo}
\begin{verse}
Na všech vrcholcích vysokých stromů\\
špačci sedí.\\
Vrzající korouhvičky dubna\\
z velmi staré mědi.
\end{verse}
\begin{verse}
Rozmrzele stříhají plachty nebe\\
šedé, neprůhledné.\\
Člověk přece před svou budkou\\
rád si na slunko sedne.
\end{verse}
\begin{verse}
Všecky vrcholky vysokých stromů\\
kolébají se s nimi.\\
Vítr rozhání s plachtami nebe\\
zbytky zimy.
\end{verse}
\begin{verse}
Hbitě na jednu usedl jsem si,\\
pluji jako v člunu.\\
Drobty slunce házím dolů\\
stromům na korunu.
\end{verse}
\begin{verse}
Stromům na korunu, na střechy dědině,\\
špačkům do hrdla rovnou.\\
Máme všichni nakonec radost\\
nevýslovnou.
\end{verse}
\newpage
\poemtitle{Stesk na jaře}
\begin{verse}
Všichni ptáci dnes mne pozdravili,\\
sojka křičela a datel zaklepal,\\
modré oči jaterníků pily\\
lačně úsměv dne, jímž lehký větřík vál.
\end{verse}
\begin{verse}
Žluťásek a babočka se chvěli,\\
mouchy sedaly na místa slunečná.\\
Šel jsem vesel, pohled můj byl smělý,\\
snad i bída má se zdála konečná\...
\end{verse}
\begin{verse}
S chladem večerním však zhasly náhle\\
bílé plaménky mé senzibility.\\
Smutku zalily je viny táhlé,\\
šerem celý svět se zdál být zalitý.
\end{verse}
\begin{verse}
Opět chuďas byl jsem živořící,\\
jenž se ničemu vzdát nesmí docela.\\
Šerem kráčel fantóm bledolící,\\
trhal poupata, jež chtivě pučela.
\end{verse}
\newpage
\poemtitle{V den větrný}
\begin{verse}
Bláznivý větře z tohoto dubna,\\
říkám si píseň o Heřmanu z Bubna\\
z teplé své sednice zíraje v dál.\\
Oblaka bílá na modrém nebi\\
jako ty myšlenky v bujné mé lebi\\
letí a letí za lesů val.
\end{verse}
\begin{verse}
Oblaka bílá a oblaka šedá\\
-- to ty jsi, větře, jenž pokoje nedá --\\
hrají si s tebou a se sluncem:\\
děvčice nezbedné, děvčice bílé\\
-- zrodila, zhltne je větrná chvíle --\\
laškují se zlatým milencem.
\end{verse}
\begin{verse}
Řekni však, větře, křídla zda mají,\\
že se z nich práší až po našem kraji,\\
když do nich pereš mladicky rád?\\
Či je to těl jejich bělostné chmýří,\\
které jim ohnivost milence víří\\
a jež k nám slétá z jich ukrytých vnad?
\end{verse}
\begin{verse}
Bláznivý větře, dubnový větře,\\
veselý cval tvůj až k srdci se vetře\\
s veselým sluncem a s oblaky.\\
Říkám si píseň o Heřmanu z Bubna,\\
směji se vločkám z tohoto dubna,\\
tvořícím marnivé povlaky.
\end{verse}
\begin{verse}
Na mladé trávě zostřují zeleň,\\
říkám si s tebou: Člověče, neleň,\\
co nám již může nějaký sníh!\\
Pučící habříky proměnil v břízky,\\
k srdci však nesmí, a květen je blízký,\\
v myšlenkách udatnost, radost a smích!
\end{verse}
\newpage
\poemtitle{Zpěv země}
\begin{verse}
Jdu podle vody do šera,\\
jež od lesů se plíží.\\
Kdes nad dědinou vyvěrá\\
a váhavě se drolí\\
klekání, lidská píseň těch,\\
kdož pod života tíží\\
sehnuti, s hlavou v ramenech\\
jdou z dílen, dvorců, polí.
\end{verse}
\begin{verse}
Jak naléhání v pozdní čas,\\
stesk bázlivý a chvějný,\\
blekotá za mnou zvonku hlas\\
ke stráním, kde se stmívá;\\
unikající víry žel\\
a nářek beznadějný\\
jak by se za mnou potácel\\
a s řeky šumem splývá.
\end{verse}
\begin{verse}
Však od splavu mně šerem vstříc\\
zní píseň jiná, smělá,\\
proud řeky hučí stále víc\\
a z lesů, luk a ze skal\\
vzkaz půdy nese vzrušené,\\
že pouta padla z těla\ldots\\
A k bouři vody zpěněné,\\
jak by tam někdo tleskal.
\end{verse}
\begin{verse}
To ve vlnách, jež do běla\\
na balvanech se tříští,\\
snad vodní žínka seděla,\\
má radost ze života\ldots\\
Jdu podle vody, u splavu\\
to skučí, řve a piští,\\
proud staví se tu na hlavu\\
a zlostně zaklokotá.
\end{verse}
\begin{verse}
Však vše, co v dáli zápolí\\
po dubinách i v boří,\\
na vrších, v žlebech, v údolí,\\
ruch v poupatech a mechu,\\
děj zeleného zázraku,\\
jenž pomalu se tvoří, --\\
já slyším z řeky v soumraku\\
a cítím v jejích dechu.
\end{verse}
\begin{verse}
Jdu podle vody v temnotu,\\
jenž nesmírně je živá,\\
rašení tajnou klopotu\\
šum řeky doprovází,\\
zpěv táhne mezi stráněmi,\\
milenka země zpívá:\\
silného láskou oněmí,\\
kdo sláb je, toho srazí.
\end{verse}
\newpage
\poemtitle{Lamači}
\begin{verse}
Ve stráni u Svitavy, ve světlé jízvě lomu\\
lámeme kámen bystře s tichými dělníky pěti;\\
s dobráckým úsměvem slunce shůry přihlíží k tomu,\\
snad vzpomínajíc, že skálu vidělo kameněti.
\end{verse}
\begin{verse}
%FIXME modrozelený bylo původně (bez "O")
O modrozelený kámen kladiva rytmicky zvoní,\\
odskakujeme vesele, o hrany klopýtáme,\\
hevery derem se do skulin, až balvan k pádu se kloní,\\
ve tváři oheň, krev na rukou, krůpěje na čele máme.
\end{verse}
\begin{verse}
Nad námi modrá nebesa, pod námi hučí řeka,\\
dubnová pohoda křehká po stráních táhne sladce,\\
mladistvá zeleň dole svítí jasný a měkká;\\
kámen a železo zpívají radostnou píseň práce.
\end{verse}
\begin{verse}
A nač jsou příliš pomalé a slabé ruce lidské,\\
to prachem trháme\ldots Pozor! Dým, rána. Letí kusy,\\
balvany s rachotem řítí se, a v snaze harmonické\\
mocněji klokotá řekla zrcadlíc plápol rusý.
\end{verse}
\begin{verse}
A večer nahoře nad lomem, kde trnkových keřů je pole\\
bílými perličkami poseto v tichém vznětu\\
a zítra bude již rozvito a potom posype dole\\
zpocená těla dělníků lehounkým sněhem květů,\\
\end{verse}
\begin{verse}
nahoře nad lomem, myslím si, zíraje na kameny:\\
Lamačů práce těžká je a hazardní a krásná,\\
lamačů práce šťastná je jak svátek osluněný, --\\
když narub života nemyslíš pro vesnu, jež jde jasná.
\end{verse}
\newpage
\poemtitle{Bouře}
\begin{verse}
Och, první májová bouře\\
mě venku zastihla včera;\\
v údolí Svitavy vpadla\\
hodina přísná a šerá.
\end{verse}
\begin{verse}
Od stráně ke stráni haslo\\
vášnivě zlomené světlo.\\
A nebe rozdoutnalo se,\\
by náhle ohnivě vzkvetlo.
\end{verse}
\begin{verse}
Zelení chvějivý příval\\
ustrnul v plastičnost chmurnou.\\
Docela zčernalo boří\\
jak s odhodlaností vzpurnou.
\end{verse}
\begin{verse}
A pak to zaválo dolem.\\
Blesky a hromy se líhly\\
v nebesích převalujících\\
dýmy, jež za zemí tíhly.
\end{verse}
\begin{verse}
Za kapkou kapka. A náhle\\
země se s oblohou střetla;\\
obloha ledový příval\\
jí ve tvář divoce vmetla.
\end{verse}
\begin{verse}
To byly kroupy. Jen okamžik.\\
Skryt špatně zřel jsem, jak bijí\ldots\\
Člověk se slunečnem jednou\\
a bouří podruhé zpíjí!
\end{verse}
\newpage
\poemtitle{V modřínech}
\begin{verse}
V květnovou zeleň modřínů,\\
ve smaragdovou lázeň\\
hodil jsem srdce únavu\\
a mozku střízlivou kázeň.
\end{verse}
\begin{verse}
Dostal jsem ukrutnou, divou chuť\\
celý ponořit se do ní;\\
ne jako člověk, jak strom se chvět\\
pod cvalem větrných koní.
\end{verse}
\begin{verse}
Tak zatraceně býti nah!\\
Tak moudrý, pevný, svěží!\\
Tak samozřejmě přijímat\\
měnlivý život, jenž běží!
\end{verse}
\begin{verse}
Celý jsem vtiskl se do stráně,\\
jí sestoupil jsem až ke dnu,\\
čekal, až zapustím kořeny\\
a větve zelené zvednu.
\end{verse}
\begin{verse}
Modřínů vlající chocholy\\
nad hlavou šuměly mi,\\
les v nitro vcházel mi pomalu\\
a všemi smysly mými.
\end{verse}
\newpage
\poemtitle{Kaštan}
\begin{verse}
Hle, dárce stínu velmi shovívavý.\\
A nadevšecko milující děti,\\
jimž úrodu svých měkce oblých snětí\\
pro hru i zdobu střásá z mocné hlavy.
\end{verse}
\begin{verse}
Má kypící a houževnaté zdraví,\\
a selsky rozložen, se nechce chvěti;\\
byť nemohl i prudce zavoněti,\\
cos hlubokého jeho červen praví.
\end{verse}
\begin{verse}
Když po západu vlažné ticho voní\\
a vlažným vzduchem slzy štěstí kanou\\
v kraj milostný, jenž cikádami zvoní,
\end{verse}
\begin{verse}
kdes v aleji či v sadě, kde se setmělo,\\
tak jeho květy bílé náhle vzplanou,\\
že dívku čekáš, jež ti nese tělo.
\end{verse}
\newpage
\poemtitle{Kukaččino volání}
\begin{verse}
Kukaččino volání do květnových lesů,\\
jaká výzva sváteční, jaké vyznání!\\
Z tisícerých pokřiků, zpěvů, hvizdů, hlesů\\
jaký povel jásavý k pomilování!
\end{verse}
\begin{verse}
Buben, zvon a polnice v jednom hlasu slity\\
sonorní své trocheje hudou z vrcholků\ldots\\
Kdybys houštím se mnou šla, zelenými svity,\\
jistě nečinila bys marných okolků.
\end{verse}
\begin{verse}
Poplach srdcí bušících, mízy útočící,\\
neúnavných hrdélek, v nichž to klokotá,\\
světlou vlnu přílivu, oddech křepkých plící,\\
pochod, útok, vítězství slávy života
\end{verse}
\begin{verse}
ze zelených cimbuří melodický hlásá,\\
až se v každém stéble trav ozve odpověď.\\
I v tom žlebu nejhlubším slyšíš, kterak jásá,\\
i v tom žlebu nejhlubším rozechvívá svět.
\end{verse}
\begin{verse}
A kdybys, má lásko, šla se mnou houštinami,\\
v shodě s květnem dala květ, na němž bych se zpil, --\\
slyšela bys v tluku tom v jasu nad hlavami\\
sladký trochej milostný, jenž nás uchvátil.
\end{verse}
\newpage
\poemtitle{Poledne v seči}
\begin{verse}
Mám plná ústa pryskyřic\\
a myslím, že utonu dnes\\
v zeleně proudech dychtivých,\\
jež ze všech stran se řítí.\\
Je poledne. Ležím u seče.\\
Do vlažné trávy jsem kles.\\
Po tichých deštích májových\\
vzduch hoří, chvěje se, svítí.
\end{verse}
\begin{verse}
A já tu ležím pokořen,\\
života bláznivým dnem,\\
jenž tryská, šílí, zpívá, žhne,\\
var pudů, hladů, žízní;\\
jsem zmámen vzruchem instinktů\\
a tvarů prchavých snem,\\
jež od země tíhnou za sluncem\\
a synteticky vyzní.
\end{verse}
\begin{verse}
Mám plná ústa pryskyřic\\
a barev plničký zrak;\\
šum promíšených pohybů\\
jak moře v sluch mi hučí:
\end{verse}
\begin{verse}
já ležím na dně, trosečník,\\
jenž opustil svůj vrak\\
a život pod hladinou teď\\
se milovati učí.
\end{verse}
\newpage
\poemtitle{Strašidla v kraji}
\begin{verse}
Jdu z města lesům vstříc a upomínka se mnou\\
na vlnky tančící pod látek smělou barvou,\\
na jaro v tělech žen, jež s ukrutností jemnou\\
pro ňadra, boky, klín si steré oči narvou;
\end{verse}
\begin{verse}
jdu z města horkého a jarní večer tmí se,\\
a hore povlává van váhavý a nyvý,\\
tma z lesů dere se a svítí cesty lysé,\\
klid svěží ovanul mé rozpálené čivy.
\end{verse}
\begin{verse}
Tu ženy bláhové, jež ve městě jsem viděl,\\
jichž rozhoupaný prs tak těžko z mysli mizí,\\
a lesů představa, jež plny trav a zřídel\\
i ptáků, vůni, tmy jsou jako lázeň ryzí,
\end{verse}
\begin{verse}
a stráně zastřené a vzhůru cesta sivá\\
a soumrak májový, jenž chvěje se a voní, --\\
to všechno mísí se a harmonicky splývá,\\
že sladkou rozkoší div prsa nezazvoní.
\end{verse}
\begin{verse}
Však náhle v temnotě pod lípou na rozcestí\\
a opět u škarpy, kde sedlák zhynul kdysi,\\
a v poli u lesa jak hrozba bílé pěsti\\
a u dědiny zas, jak sebevrah když visí,
\end{verse}
\begin{verse}
smrt, vraždu, neštěstí tak připomínajíce\\
se tyčí kameny a chmurné kusy kovu\\
jak velké, pitvorné a ztvrdlé neštovice,\\
jak úskok, nepřítel, jenž připraven je k lovu,
\end{verse}
\begin{verse}
jak hnusná strašidla, jak ženy nakažené,\\
jež z temna neřestně tě chtějí osloviti,\\
jak vrazi silniční, jimž v ruce napřazené\\
nůž dobře broušený a zkrvácený svítí\ldots
\end{verse}
\begin{verse}
Tak ve kraj lahodný, kde země sladce dýše,\\
kde člověk rozkoše jen sobě připomíná,\\
a v soumrak vášnivý, jenž jako měkká skrýše\\
pro lásku hlubokou svá hebká křídla spíná,
\end{verse}
\begin{verse}
ty chmurné symboly a těla teskně mroucí\\
jen smutek vnášejí a souzvuk věd plaší,\\
na paměť stojíce prý lásce nad vše vroucí\\
jen lásku drsností svou na rozcestích straší.
\end{verse}
\newpage
\poemtitle{Střevlíci}
\begin{verse}
Zmolinou lesní úzkou, příval vypral ji včera,\\
kameny umyl v ní a schody nanesl z písku,\\
v stříbrném vzduchu stoupám do zeleného šera\\
vzhůru a vzhůru, líbám habr, smrček i lísku\\
pohledem vděčným za všecko.
\end{verse}
\begin{verse}
Divím se srdcím šťavelu, hlemýžďům bez ulitky,\\
naslouchám ptákům, sbírám brouky, laskám se s kvítky\\
a mám se za děcko.
\end{verse}
\begin{verse}
Zmolinou lesní stoupám, a legion očí svítících,\\
kamkoli hnu se, hledí na mne a zrcadlí smích\\
zelení, jež se blahem dmou z vláhy po dnech sucha.\\
Je ticho pod klenbou; ze samých smaragdů je složena,\\
černé a hnědé sloupy ji nesou, černá a hnědá ramena,\\
a já mám volného ducha\ldots
\end{verse}
\begin{verse}
Ale teď. Hola! U nohou mých! Cože to chřestí\\
v tom tichu stráně, jež vláhou nasycena\\
unyle odpočívá jak pomilovaná žena?\\
Jako když skřítků ostřičné zbraně zašelestí\\
v souboji drobném\\
a zdobném.
\end{verse}
\begin{verse}
Hej, hola! Jsou to dva modří střevlíci,\\
po písku, jehličí, kamenech dolů se valící.\\
Hej, hola! A ty jsi to, lásko,\\
zlá, nesvědomitá plavovlásko,\\
která je ženeš objaté po kamenném loži\\
a hloží!
\end{verse}
\begin{verse}
Jaké to divadla ve zmoli\\
na lesní stráni!\\
Samička prchá a zápolí,\\
šesterem stehen se brání.
\end{verse}
\begin{verse}
Sameček slepý ze všech sil\\
za krk ji drží, za kolena.\\
Tvůj, lásko, dech mu rozpálil\\
krunýř až do zelena.
\end{verse}
\begin{verse}
Hned dole je, hned navrchu,\\
a kovově chřestí jim tělo.\\
Ze svraštělého povrchu\\
teď jak by to zasršelo.
\end{verse}
\begin{verse}
A teď zas v pevném objetí\\
se z kamene na kámen řítí.\\
Och, jaké dnes tvé zakletí\\
jim schystalo živobytí!
\end{verse}
\begin{verse}
Ale již vítězný sameček\\
pod sebou drží ji směle,\\
a jeho bílý údeček\\
zdvíhá se, chvěje se vřele,\\
dobývá skuliny\\
pro sladké vteřiny --\\
och, lásko!
\end{verse}
\begin{verse}
Tak tedy tě, lásko, potkávám ne zmolině lesní,\\
na tvrdých kamenech, na písku sneseném,\\
a kol nás jaro svlažené a ničeho, co teskní,\\
ledaže divoženka v úkrytu zeleném\\
snad lačně, snad zvědavě mým trápí se člověčenstvím,\\
jako já ženstvím.
\end{verse}
\begin{verse}
Je ticho pod klenbou, rozpjatou smaragdově,\\
jen ptáků hnízdících varovné hvízdají křiky!\\
Stříbrným vzduchem vonným, duchu i tělu hově,\\
za voláním kukačky jdu, za zlatými pryskyřníky\\
v aleje spanilé\\
bez cíle\ldots
\end{verse}
\newpage
\poemtitle{Dedikace}
\begin{verse}
Jahody první, hle, nesu ti, ženo,\\
z bukové paseky, z vysoké trávy,\\
pozdravy slunce a jižního větru,\\
zakleté v šarlat a ve sladké šťávy:
\end{verse}
\begin{verse}
Pomalu nes je k svým hedvábným retům,\\
ať jejich sladkost až k srdci ti vzrůstá.\\
Pomni, že políbíš červnové slunce\\
a krví země si potřísníš ústa.
\end{verse}
\begin{verse}
Jahody první, hle, nesu ti z lesů,\\
vyhřátých paprsků úrodou zlatou.\\
%buď místo bud
Jak ten plod rudý buď krystalem vonným,\\
%buď místo bud
slunce a země buď pohodou jatou.
\end{verse}
\newpage
\poemtitle{Chvála nahoty}
\begin{verse}
Nad lesy červen plá a nebe safírové,\\
však pod korunami jen ticha šumí hlasy;\\
jsou květy nachové, kde paví oko plove\\
a syto sladkostí mdle s okvětími hrá si.
\end{verse}
\begin{verse}
Den ohni zlatými sluj smaragdovou třísní\\
a šerem svatební dech vegetace vane;\\
od země plíží se pach vlhce černých plísní\\
a z nebes oštěpy v stín vůně boří kane.
\end{verse}
\begin{verse}
Nad lesy červen plá a ticho vášní voní\ldots\\
Hle, z němé nádrže teď chochol tryskl vodní\\
a vrací krůpěje, jež o hladinu zvoní;\\
proud čirý, stříbrný se rozlil v lesy spodní.
\end{verse}
\begin{verse}
Však není fontán to, ni ptačí hlas to není.\\
A hasne v korunách již sladká melodie.\\
Jen ticho chvěje se ve žhavém roznícení\\
a stébla kloní se: Což Veliký Pán žije?
\end{verse}
\begin{verse}
Snad vyšel s píšťalou na stráně hledat panny.\\
Je lačná vteřina a červen mocně velí.\\
Snad dívka vyšla si a hlas má vytepaný\\
svou touhou jako šperk, jenž zajal plamen skvělý\ldots
\end{verse}
\begin{verse}
Och, možno aspoň snít ve věku běd a muky,\\
že dívka vyšla si a náhle stála nahá\\
tam dole v pasece, kde pod starými buky\\
je tráva vysoká a laskavá a vlahá?
\end{verse}
\begin{verse}
Je možno aspoň snít o sladkém těstě v kvasu,\\
o krásné nahotě, jež lesům vpospas dána,\\
by souzvuk doplněn byl snopem plavých jasů,\\
v němž látka zápalná pro naše smysly stkána?
\end{verse}
\begin{verse}
Nic není krásnější než toto kladné dílo,\\
jež s cetkou nepadá, nemizí se závojem;\\
v něm na sta souhvězdí a jitřních střel se slilo\\
ve fontán zářných gest a prazákladní pojem.
\end{verse}
\begin{verse}
Je možno pochopit, že mýtus šperků zmate,\\
že napověděné a poloskryté dráždí.\\
Jsou lesti ničemné, jsou úskočnosti svaté.\\
Pro modlu zastřenou se člověk lehce vraždí.
\end{verse}
\begin{verse}
Však ženy nahota je život sám, ne modla,\\
blesk ozařující směr z abstrakcí a stínů!\\
S nebesy země se v úsměvu jejím shodla,\\
jenž mimo zásluhu a mimo všechnu vinu.
\end{verse}
\begin{verse}
Když křídla rozepne, žár zkojíš jejich žárem;\\
s jich spánkem hedvábným tě hudba zpije svěží,\\
jak lázeň pod strání a okouzlení tvarem,\\
když kolem jara sen a třešní květy sněží -- -- --
\end{verse}
\begin{verse}
Och, možno aspoň snít v den červnového kouzla,\\
že dívka vyšla si a zjevila se nahá,\\
by v lože vlahých trav jak nebes oštěp sklouzla,\\
dar přepozitivní a kořist nad vše drahá,
\end{verse}
\begin{verse}
do loktů mužových, jenž o hmotě pěl vzkvetlé,\\
jež zraje na slunci a z půdy šťávy saje,\\
o náboženství svém, jež konkrétní a světlé\\
bez marných abstrakcí stem žhavých barev hraje?
\end{verse}
\newpage
\poemtitle{Lípa}
\begin{verse}
Dech síly jaksi neuvědomělé\\
a dobráctví, jež medu plno, vane\\
z koruny její hmyzem opěvané,\\
když nad snem květů pohoda se klene.
\end{verse}
\begin{verse}
V jich vůni zlatem nebes nasycené,\\
s níž pozdrav slunce dolů ve stín kane,\\
mdlý poutník tuší kraje požehnané\\
a kvetoucí kdes v budoucnosti sněné.
\end{verse}
\begin{verse}
Až sladká moudrost vejde v srdce lidská,\\
jak v úly med ze sterých snesen květů,\\
a práce bude radost harmonická,\\
jak pohanské a slavné kolonády\\
lip aleje ve chvílích velkých vznětů\\
tančící páry přilákají rády.
\end{verse}
\newpage
\poemtitle{Autoportrét v červnu}
\begin{verse}
Nad lesy blankyt zapálen, stráněmi táhne dech boří.\\
Na mechu ležím v dubině já pod keřem na výsluní.\\
Na pohozeném klobouku kytice jahod mi hoří:\\
z města jsem člověk ubohý, však venkovskou mám již vůni.
\end{verse}
\begin{verse}
Z košile rozhalené krk nahý a snědý mi svítí\\
a prsa lehce omšelá temnými kučerami.\\
V mé tváři trochu vyžilé hnědé oči se vznítí,\\
kdykoli políbí mě van proudící hlubinami.
\end{verse}
\begin{verse}
Mám hubené ruce beze stop těžké fyzické píle,\\
však hlínou a šťávami bylin hojně poskvrněné.\\
Cos jistě by mluvilo k ženě teď o mé smyslné síle,\\
přívalem vlažných dojmů příjemně obelstěné.
\end{verse}
\begin{verse}
Rukou si podpírám hlavu pod černou zcuchanou kšticí\\
a hledím zdlouhavě, něžně, jako bych hladil v lese\\
květiny, trávy, kmeny, balvany v mechu spící\\
a vše, co živého někde mihne a zaleskne se.
\end{verse}
\begin{verse}
Kniha mi vyhlédá z kapsy kabátu pomačkaného.\\
Však připoután k zemi je tu subtilní sluch lidský:\\
člověk a hrouda si sdělují rytmus nitra svého,\\
žár krve, píseň zrání, stesk temně erotický.
\end{verse}
\newpage
\poemtitle{Polemika}
\begin{verse}
Ne, nepůjdu do Říma dotknout se antik svou neuctivou rukou;\\
mám nad hlavou slunce, jež miluji,\\
s ním lesy a vrchy mě stvrzují,\\
co krásou, silou, štěstím života, co bídou a mukou.
\end{verse}
\begin{verse}
Tam velebně Via Appia hlásá život, jenž zapadl v tůně;\\
zde prostě a pravdivě zpívají\\
%FIXME nebo šťávách?
o života šťavách, jež tryskají,\\
ptáci a vody, hmyz a louky, linie, barvy, vůně.
\end{verse}
\begin{verse}
Tam na troskách slávy kraluje hlupák a hladová bída lidu;\\
zde živoucí krásy a naděje,\\
zvěř, květiny, mechy a ručeje\\
pohanské pudy chválí písni majestátního klidu.
\end{verse}
\begin{verse}
Zde jasně cítím, jak malí jsou bozi ti staří i ti noví\\
a člověk jak mocně se vztyčuje;\\
když se zemí vroucně se miluje\\
a její nápěv doprovází chvějícími se slovy.
\end{verse}
\newpage
\poemtitle{Improvizace}
\begin{verse}
Mateřídouškou ruce mi voní;\\
ležel jsem na mezi, naslouchak skřivanu,\\
zda je to pták, či zda srdce to zvoní\\
mně po ránu.\\
Procitlá země, nadechlý sen,\\
slibovala mi rozkošný den.\\
\end{verse}
\begin{verse}
%FIXME nový verš jsem sem přidal sám, vyplývá z toho,
%že každý jiný verš začíná o vůni rukou
A jahodami ruce mi voní,\\
mám krev jejich na rukou, na retech.\\
Přišel jsem do seče, plazil se po ní,\\
na slunci poledním vřel její dech;\\
padl jsem hluboko do trávy\\
v den jásavý.
\end{verse}
\begin{verse}
A borem ruce mi voní.\\
Po stráni dolů běžel jsem bořinou,\\
kmenů se chytal, těchže jak loni,\\
sklouzal jsem, klopýtal za srnčí rodinou.\\
Opilý lesy, divoký, nevinný\\
stanul jsem nad domky dědiny.
\end{verse}
\begin{verse}
A ještě něčím ruce mi voní;\\
Svou milenku našel jsem pod jabloní.
\end{verse}
\newpage
\poemtitle{Genesis}
\begin{verse}
Já, zdá se, narodil se v Městě krutém,\\
jež nenáviděje hned opět miluji,\\
já, zdá se, hýřil v plynů světle žlutém\\
a mrzce s ženami, jež hnusem nadují.
\end{verse}
\begin{verse}
Však sladce zapomněno všechno bývá\\
ve žlebu provlhlém a hnědě hnijícím,\\
když odtud na vrchy mě zeleň živá\\
jak svého synáčka zve jasným ohněm svým.
\end{verse}
\begin{verse}
Tu vím, že zrodil jsem se v dubin stínu\\
a mlékem trpkým laň mě v seči kojila:\\
řád lesů děl mou zásluhu i vinu,\\
s jich něhou tvrdost jich se ve mně spojila.
\end{verse}
\begin{verse}
Pak snil jsem život měst a černých čtvrtí,\\
grog horký v přístavech jsem s námořníky pil,\\
až odešel jsem náhle smuten k smrti\\
bez cíle, tlumoku, bych nyní teprv žil!
\end{verse}
\begin{verse}
Ze snů a mrákot probral jsem se v zoři,\\
jež mízou voněla od lesů vanoucí;\\
zelené vlny sladkovodních moří\\
šuměly osud můj a cesty budoucí.
\end{verse}
\begin{verse}
Tu trochu klopýtal jsem v měkkém mechu\\
omámen vůněmi a ptáků jásotem;\\
však pryskyřice dodaly mi dechu:\\
se zemí srůstaje svým žiji životem.
\end{verse}
\end{document}
